% ----------------------------------------------------------
\chapter{Related Work}\label{cap:related_works}
% ---------------------------------------------------------- Citar que uma parte dos trabalhos relacionados foram selecionados a partir de RSLs e mais alguns foram selecionado como compleme.

% Citar que muitos dos trabalhos vieram das RSLs
% 

This chapter presents the related work relevant to this research. Part of the them came from the SRLs and others added as complement by the researcher.

\section{Related works in Text Representation}


\section{Related works in Text Classification}


\section{Related works in Text Regression}


\section{Conclusions for the Chapter}

% Aqui fazer uma síntese colocando o trabalho em relação à literatura. O que ele faz que não existe na literatura.

% Classificação é mais explorado pelo fato de ser mais ser mais simples enxergar problemas nessa tarefa. Há então a aplicação e avaliação de ML numa aplicação específica que é o JEC e transporte aéreo.

% Já do ponto de vista de regressão, se aplica no direito onde também ninguém resolveu explorar, tem também a questão da avaliação dos passos da pipeline.

% Já em relação a representação, tem a parte criação de representação para o direito brasileiro que não havia sido explorado com mais aprofundamento.

% Tem também a questão de a classificação ser mais branda em termos de resultado, uma vez que diz um sim ou não. Já a regressão diz um resultado específico com uma margem de erro.

