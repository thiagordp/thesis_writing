% ----------------------------------------------------------
\chapter{Introduction}
% ----------------------------------------------------------




% \section{Motivation and context}

% Context
According to the last report \textit{Justiça em Números}, published annually by the National Council of Justice (CNJ), by the end of 2019, there was around 77,1 million ongoing processes waiting for a solution in the Brazilian Judiciary. In total, in 2019, 30,2 million lawsuits were filled in all Judiciary, an increase of 6,8\% in relation to 2018. From those processes, about 5,2 million were filled in the Special Civel Courts (JECs)  \cite{CNJ2020}. 

% Write about JECs
JECS are Judiciary bodies regulated by Law no. 9,099/1995, which seek to facilitate citizens' access to Justice through simpler and cost-free procedures. As a result, JECs tend to approach the legal problems of ordinary people who find themselves involved in daily conflicts of small economic expression, whether in the purchases they make, in the services they hire or in the accidents they suffer \cite{Watanabe1985}.

% The increasing interest in applying legal judgements (cite RSL?)
Considering the challenges faced in the judiciary system in Brazil and around the world (CITE), there is an increasing interest (CITE) of the literature on applying Artificial Intelligence (AI) based techniques to solve issues from the lower courts (CITE), such as JECs, to the superior courts (CITE).

% What is AI?
AI, according to \textcite{Russell2020}, relates to the study of intelligent agents that receives percepts or inputs from the environment and make actions. The agent uses a function to map the percepts to the appropriate actions and such function can be explicitly defined and/or learned using Machine Learning (ML) techniques (CITE).

% ML tasks in the legal environment

\todo{Improve} In the legal context, the agent receives as percept (inputs) a the case's  textual data and acts (output) setting a decision. Considering the complexity of legal judgments, such decision may be learned using ML and Text Mining (TM) techniques, \todo{Define TM}.

% What is Classification, clustering and regression in this work?
Regarding the tasks to apply in legal decisions include clustering, which is ..., classification..., regression, ... (?)

% The many applications of TM and NLP
TM techniques have been used to solve several tasks regarding textual data, such as classification (CITE DEF), regression (CITE DEF) and clustering (CITE DEF) in several contexts, such as the legal (), medical (), engineering and social media

% NLP

\todo{Continue...}


% ----------------------------------------------------------
\section{Problem Definition} % Problem and Hypothesis
% ----------------------------------------------------------

% The importance of the JECS and the problem of 
The judicial lawsuits processed at the JECs are decided manually by the judge, and there is no automation in that sense. This leads to slowness and, as a consequence, the large number of processes pending solution (CITE). In addition, these processes are composed of unstructured textual data that, in addition to being represented in natural language, have their own legal vocabulary (CITE). 

% falar da RSL que foi feita e que vamos detalhar depois)

% 

\section{Research Question}

``Is it possible to predict the result of a legal case based on its content and predict the amount of compensation for immaterial damage using machine learning and text mining techniques?''

The question can be broke down into three:

\begin{itemize}[noitemsep]
    \item How to translate the complexity of the legal language to a numerical representation?
    \item Which machine learning techniques for classification can bring an legally acceptable accuracy to the predicted lawsuit result?
    \item Which machine learning techniques for regression can bring an legally acceptable error to the predicted amount of compensation for immaterial damage?
\end{itemize}

% ----------------------------------------------------------
\section{Objectives}
% ----------------------------------------------------------

In this section we introduce to the reader the main objective and the specific objectives necessary to achieve it.

% ----------------------------------------------------------
\subsection{Main Objective}
% ----------------------------------------------------------

To evaluate if we can predict with a legally acceptable amount of accuracy the result of a legal case and predict the amount of compensation using machine learning and text mining techniques.

% ----------------------------------------------------------
\subsection{Specific Objectives}
% ----------------------------------------------------------

To do that we need to:

\begin{itemize}[noitemsep]
    \item Demonstrate that representing the legal cases numerically using word embeddings and BOW can achieve legally acceptable results in the classification and regression tasks.
    \item Demonstrate that it is possible to predict the lawsuit result using classical and deep machine learning techniques for classification with legally acceptable accuracy.
    \item Demonstrate that it is possible to predict the amount of compensation  using classical and deep machine learning  techniques for regression with legally acceptable error.
\end{itemize}

% ----------------------------------------------------------
\section{Justification and Subject Relevance}
% ----------------------------------------------------------
\begin{itemize}[noitemsep]
    \item RSL Results
    \item AI Regulations and initiatives
    \item Growth in interest 
\end{itemize}

% ----------------------------------------------------------
\section{Methodological Procedures}
% ----------------------------------------------------------

\begin{itemize}[noitemsep]
    \item Systematic Review
    \item Legal Dataset 
    \item Experiments using Python
    \item Evaluation
    \item Check Legal Acceptable Accuracy and Errors
    \item Resources
\end{itemize}

% Structure
% Include here the research delimitation
% The steps we have followed to answer the question


% ----------------------------------------------------------

\section{Contributions}
% ----------------------------------------------------------

During the research, experiments were conducted to answer the research questions. Some of these experiments resulted in publications in journals and conferences. Considering the published works and the experiments applied, the contributions of these work are as follows:

\begin{itemize}[noitemsep]
    \item Pre-trained word embeddings models for Brazilian legal texts, since there was no available representation available before.
    \item Impact of adjustments in the pipeline for regression and classification.
    \item As real life application, we would help the Judiciary by helping to end the lawsuits in JEC at the conciliation hearing step.
\end{itemize}

% Contribuições mais acadêmicas
% Contribuições mais práticas
% - Help to speed up the process by ending them at the conciliation hearing.

% ----------------------------------------------------------
\section{Document Organization}
% ----------------------------------------------------------

The work is structured in X chapters, beginning with this introduction.

In \citechapter{cap:ml_text}, we introduce...

In \citechapter{cap:related_works}, ...







% ============================================================================================ %







% As orientações aqui apresentadas são baseadas em um conjunto de normas elaboradas pela \gls{ABNT}. Além das normas técnicas, a Biblioteca também elaborou uma série de tutoriais, guias, \textit{templates} os quais estão disponíveis em seu site, no endereço \url{http://portal.bu.ufsc.br/normalizacao/}.

% Paralelamente ao uso deste \textit{template} recomenda-se que seja utilizado o \textbf{Tutorial de Trabalhos Acadêmicos} (disponível neste link \url{https://repositorio.ufsc.br/handle/123456789/180829}) e/ou que o discente \textbf{participe das capacitações oferecidas da Biblioteca Universitária da UFSC}.

% Este \textit{template} está configurado apenas para a impressão utilizando o anverso das folhas, caso você queira imprimir usando a frente e o verso, acrescente a opção \textit{openright} e mude de \textit{oneside} para \textit{twoside} nas configurações da classe \textit{abntex2} no início do arquivo principal \textit{main.tex} \cite{abntex2classe}.

% Conforme a \href{https://repositorio.ufsc.br/bitstream/handle/123456789/197121/RN46.2019.pdf?sequence=1&isAllowed=y}{Resolução NORMATIVA nº 46/2019/CPG} as dissertações e teses não serão mais entregues em formato impresso na Biblioteca Universitária. Consulte o Repositório Institucional da UFSC ou sua Secretaria de Pós Graduação sobre os procedimentos para a entrega. 

% \nocite{NBR6023:2002}
% \nocite{NBR6027:2012}
% \nocite{NBR6028:2003}
% \nocite{NBR10520:2002}

% % ----------------------------------------------------------
% \section{Objetivos}
% % ----------------------------------------------------------

% Nas seções abaixo estão descritos o objetivo geral e os objetivos 
% específicos.

% % ----------------------------------------------------------
% \subsection{Objetivo Geral}
% % ----------------------------------------------------------

% Descrição...

% % ----------------------------------------------------------
% \subsection{Objetivos Específicos}
% % ----------------------------------------------------------

% Descrição...