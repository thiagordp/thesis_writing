% ----------------------------------------------------------
\chapter{Introduction}
% ----------------------------------------------------------

% \section{Motivation and context}

% Context
According to the last report \textit{Justiça em Números}, published annually by the National Council of Justice (CNJ), by the end of 2019, there was around 77,1 million ongoing processes waiting for a solution in the Brazilian Judiciary. In total, in 2019, 30,2 million lawsuits were filled in all Judiciary, an increase of 6,8\% in relation to 2018. From those processes, about 5,2 million were filled in the Special Civel Courts (JECs)  \cite{CNJ2020}. 

% Write about JECs
JECS are Judiciary bodies regulated by Law no. 9,099/1995, which seek to facilitate citizens' access to Justice through simpler and cost-free procedures. As a result, JECs tend to approach the legal problems of ordinary people who find themselves involved in daily conflicts of small economic expression, whether in the purchases they make, in the services they hire or in the accidents they suffer \cite{Watanabe1985}.

% The increasing interest in applying legal judgements (cite RSL?)
To address the challenges faced in the judiciary systems in Brazil and around the world \cite{Sadiku2020, }, there is an increasing interest of the literature,  on applying Machine Learning (ML) and Text Mining (TM) based techniques in the legal area (see Systemmatic Review of the Literature (SRL) in Appendix \ref{ap:rsl_ml_law}).  The papers try to solve a variety of legal problems, such as the information extraction from contracts \cite{Hassan2020}, and the prediction of decisions in lower courts and superior courts \cite{Sulea2017,  Virtucio2018}. 

% What is ML?
According to \textcite{Mitchell1997}, ML focuses on constructing computer systems that learn through experience. Thus, such systems can learn to classify texts, control robots, predict weather and others \cite{Sebastiani2002, Kober2013, Shi2015}. ML based systems can learn using some approaches \cite{Schmidhuber2015, Caruana1997}, including supervised and unsupervised. In supervised ML, the system maps a relationship between inputs and outputs based on \textit{a priori} knowledge from labed data. In unsupervised ML, the system tries to find patterns in the data considering the similarities among the many date points \cite{Theodoridis2009}.

TM relates, according to \textcite{Aggarwal2013},  to the idea of discovering and analyzing patterns, such as trends and outliers, from textual data. TM also focus on helping users to analyze and digest information towards a better decision making. Thus, the necessity of TM and ML based applications emerges with the large amount of textual data  generated by users, companies, universities and so on, and human limitations to analyze it \cite{Lecun2015, Khan2014}. 

% What is Classification, clustering and regression in this work?
TM and ML have been used to address challenges regarding supervised learning, such as text classification,  text regression, and unsupervised learning like text clustering \cite{Aggarwal2013, Trusov2016, Medvedeva2019, Zhang2019}. 



% ----------------------------------------------------------
\section{Problem Definition} % Problem and Hypothesis
% ----------------------------------------------------------


% Mas os casos são decididos manualmente pelo juiz, o que pode gerar delay.

% Alta judicialização
Considering the high level of judicialization in Brazil, where the judiciary is increasingly assigned to solve day-to-day conflict, less cases are decided with agreements between the parties. So they shall wait for a final judgment, which may take from days to years \cite{Cury2019, Mancuso2020}.


% Talvez se as partes tiverem alguma ideia da tendência de decisões em casos como o seu, eles podem fechar um acordo antes e reduzir o tempo de espera.
If the parties had more information on the possible outcomes, based on the judge's previous decisions, they would achieve a consensus and finish the case, without the need of waiting for a judgment. In this context, the use of ML and TM can help in the conciliation between the parties by predicting the possible outcomes based on the previous decisions.  


% Muito coisa pro juiz decidir
% Mas os casos são decididos manualmente pelo juiz, o que pode gerar delay.
In the  JECs, the cases are decided manually by the judge, and there is no automation in that sense. Thus, this leads to slowness and, as a consequence, an impact on the large number of processes pending solution \cite{CNJ2020}. 

% Melhorar
Moreover, depending on the legal contexts,  the judge sets an amount of compensation to be paid by a party to the other. However, there is no explicit rules to define them.  Thus, the amount of compensation depends on the judge's interpretation of the case \cite{Sadiku2020}. 


Thus, this work intends to make contributions to the state of the art of ML and TM applied to legal texts by investigating possible solutions to the prediction of legal cases outcomes.
To get such achievement, three challenges are addressed. First, the legal text, its  complexity, and vocabulary must be represented numerically allowing the use of ML techniques. Second, the prediction of the chances of winning or not a case. And third, the prediction of the amount of compensation when the party wins the case. 

% Continuar aqui

% Mas há poucos trabalhos relacionados ao direito brasileiro em termos de classificação, poucos que abordam o problema de representação e nenhum abordando a regressão conforme a RSL que fizemos.

The first challenge is addressed in several languages, like Polish, English and Chinese \cite{Chalkidis2019, SmywiskiPohl2019}. However, from a SRL (see Appendix \ref{ap:rsl_representation_law}), no work explored the representations of legal texts in the Portuguese language using techniques like word embeddings, BERT and others.

The second challenge relates to classification, a supervised ML task which tries to learn a model to map a set of records to one of a set of labels \cite{Aggarwal2013}. Based on a SRL (see Appendix \ref{ap:rsl_ml_law}), the classification task has been applied in many legal contexts (CITE 4). However, such work, do not (FINISH, after checking the SRL).

The third challenge can be addressed with the regression task, an supervised ML task which aims to train a model to map the relationship between an input $x$ to a continuous output $y$ \cite{Draper1998}. According to a SRL (See Appendix \ref{ap:rsl_regression_law}), to the best of our knowledge, it is not addressed in any work in the literature. Thus, there is the possibility of many contributions in this regard.


% falar da RSL que foi feita e que vamos detalhar depois)

% 



\section{Research Question}

``Is it possible to predict the result of a legal case based on its content and predict the amount of compensation for immaterial damage using machine learning and text mining techniques?''

The question can be broke down into three:

\begin{itemize}[noitemsep]
    \item Which representation techniques can translate the complexity of the legal language to a numerical representation to achieve legal acceptable results in ML tasks?
    \item Which machine learning techniques for classification can bring an legally acceptable accuracy to the predicted lawsuit result?
    \item Which machine learning techniques for regression can bring an legally acceptable error to the predicted amount of compensation for immaterial damage?
\end{itemize}

% ----------------------------------------------------------
\section{Objectives}
% ----------------------------------------------------------

In this section we introduce to the reader the main objective and the specific objectives necessary to achieve it.

% ----------------------------------------------------------
\subsection{Main Objective}
% ----------------------------------------------------------

To evaluate whether we can predict with a legally acceptable amount of accuracy the result of a legal case and predict the amount of compensation using Machine Learning and Text Mining techniques.

% ----------------------------------------------------------
\subsection{Specific Objectives}
% ----------------------------------------------------------

In order to fulfill the main objective, the following specific objectives must be achieved:

\begin{itemize}[noitemsep]
    \item Evaluate whether representing the legal cases numerically using word embeddings and BOW can achieve legally acceptable results in the classification and regression tasks.
    \item Evaluate whether it is possible to predict the lawsuit result using classical and deep machine learning techniques for classification with legally acceptable accuracy.
    \item Evaluate whether it is possible to predict the amount of compensation  using classical and deep machine learning  techniques for regression with legally acceptable error.
\end{itemize}

% ----------------------------------------------------------
\section{Justification and Subject Relevance}
% ----------------------------------------------------------

% Relevancia do tema.
There is an increasing interest in the literature on the applications of ML and TM techniques in the legal domain. From that, the concerns about the implications of AI uses in the legal domain and others emerge \cite{Braz2018a, Davis2020}. Worldwide, such concerns provoked the debate of regulations on AI \cite{Cath2018}. In 2020, the National Council of Justice (CNJ) created an ordinance to regulate the uses of AI in the Brazilian Judiciary \cite{CNJ2020b}. Thus, the application of ML in the legal domain has important theoretical and practical relevance.

% Como justificamos a pesquisa.
Through the construction of three SRLs, detailed in Appendix \ref{ap:rsl_ml_law}, \ref{ap:rsl_representation_law} and \ref{ap:rsl_regression_law}, we observed such increasing research interest on the application of ML and TM in the legal domain in the last years. With the first SRL, we focused on the literature regarding ML and TM in the legal domain. As a result, most of the selected work related to the predictions in legal documents in many languages. However, until the SRL search date, none of the papers address as many classification ML techniques on the same dataset as proposed in this work. 

% Falar sobre a facilidade de se fazer experimentos com classificação e por isso a grande quantidade de trabalhos e pouco a contribuir nessa parte.

In terms of representation of legal texts, this SRL showed us some relevant work. However,  to address the first part of our research question, more research was needed. To serve as a complement, the second SRL focus on the representation of legal texts, and the results showed us that many representation techniques have been explored in the legal domain, including word embeddings and Bidirectional Encoder Representations from Transformers (BERT). However, the available work involving  the Portuguese language did not focus on the aspects of representations, that is, just using the existing representations as tools, until the SRL search date.

The third SRL focused on the application of regression in legal texts. However, such research did not find any work in that matter, although the use of many search keywords patterns. A broader literature search, detailed in Appendix \ref{ap:rsl_regression_law}, showed work regarding other contexts, including Financial and Health. Thus, this work's main contributions relates to the use of regression in legal texts with the focus on predicting compensations.

In terms of theoretical relevance, this work brings  contributions in terms of representation of legal texts as it trains representation techniques in Brazilian legal documents to be applied in supervised learning tasks. In terms of predicting the legal case result through classification, the  amount of published work limits our contributions to the evaluation on how the classical and deep learning techniques behave when applied to cases from JECs. In terms of predicting the amount of compensation through regression, this work bring contributions on impact of several Text Mining and ML techniques in the pipeline as the performance of the ML models in such task.

In terms of practical contributions, the models and pipelines from this work can be applied in real legal conciliation hearings in the JEC at UFSC. A legal expert would present and explain the predictions to the parties. Thus, we expect to help them on reaching an agreement without the need of waiting for a judgment. In this way, the litigation in the JEC would decrease, contributing to faster and more efficient access for citizens to justice.

% ----------------------------------------------------------
\section{Research Method and Delimitation}
% ----------------------------------------------------------

% O método é um conjunto de passos necessários para demonstrar que o objetivo proposto foi atingido. Se os passos do método forem seguidos, os resultados devem ser convincentes
% 

To answer the main question and its three pieces, the researcher followed a set of steps. Such pieces share common steps.

As the first step, prior to any question definition, is the construction of a Systematic Review of the Literature to evaluate the State of the Art of the application of Machine Leaning and Text Mining in the Legal Domain. Two additional SRLs complemented the first as they added specific information on the research of text representation for machine learning tasks and the application of regression in legal texts. From this step, the question and objective were set.

The second step relates to the collection of the textual datasets required to answer each question. As detailed in Chapters X, Y and Z, such datasets include an unlabeled dataset of legal cases from STF, STJ and TJ-SC and legal cases from JEC at UFSC. They also include a set of attributes extracted by a legal expert.

The third step relates to the setup and execution of machine learning experiments according to each part of the question.

The fourth step relates to the evaluation of the achieved  results in each experiment.

In terms of resources, this thesis required the use of a high performance computer from the EGOV group at UFSC and a set of open source libraries for ML.

% Talvez aqui faça sentido deixar o método mais geral (em termos de caracterizacao) e detalhar nos capítulos.

% Structure
% Include here the research delimitation
% The steps we have followed to answer the question

% Os passos pra sair do objetivo geral e chegar nos resultados
% Coleta de dados
% Pensar bem, talvez nem precise?
% Como incluir os tres problemas numa única metodologia?

% ----------------------------------------------------------
\section{Document Organization}
% ----------------------------------------------------------

% Describe why each chapter exists. What does it propose to do?
% To achieve the main goal

The work is structured in six chapters, beginning with this introduction.

In \citechapter{cap:ml_text}, we introduce the concepts of Machine Learning applied to text and the base theory to understand the three sub-problems of this work and the techniques we apply.
In \citechapter{cap:related_works}, we show the relevant  work related to each sub-problem.  In \citechapter{cap:proposal}, we show the pipelines and techniques used to answer the research questions In \citechapter{cap:results_discussion}, ...
In \citechapter{cap:final_remarks}...

% ----------------------------------------------------------

% \section{Contributions}
% % ----------------------------------------------------------

% During the research, experiments were conducted to answer the research questions. Some of these experiments resulted in publications in journals and conferences. 
% % Contribuições mais acadêmicas
% Considering the published work and the experiments applied, the contributions to the state of the art in ML applied to legal texts are as follows:

% \begin{itemize}[noitemsep]
%     \item Pre-trained word embeddings models for Brazilian legal texts, since there was no available representation available before.
%     \item New application of regression in legal textual data.
%     \item Impact of adjustments in the pipeline for regression and classification.
    
% \end{itemize}


% % Contribuições mais práticas
% In terms of practical contributions, the models and pipelines from this work will be applied in real legal conciliation hearings in the JEC at UFSC. A legal expert will present and explain the predictions to the parties. Thus, we expect to help them on reaching an agreement without the need of waiting for a judgment. In this way, the litigation in the JEC would decrease, contributing to faster and more efficient access for citizens to justice.




% ============================================================================================ %







% As orientações aqui apresentadas são baseadas em um conjunto de normas elaboradas pela \gls{ABNT}. Além das normas técnicas, a Biblioteca também elaborou uma série de tutoriais, guias, \textit{templates} os quais estão disponíveis em seu site, no endereço \url{http://portal.bu.ufsc.br/normalizacao/}.

% Paralelamente ao uso deste \textit{template} recomenda-se que seja utilizado o \textbf{Tutorial de Trabalhos Acadêmicos} (disponível neste link \url{https://repositorio.ufsc.br/handle/123456789/180829}) e/ou que o discente \textbf{participe das capacitações oferecidas da Biblioteca Universitária da UFSC}.

% Este \textit{template} está configurado apenas para a impressão utilizando o anverso das folhas, caso você queira imprimir usando a frente e o verso, acrescente a opção \textit{openright} e mude de \textit{oneside} para \textit{twoside} nas configurações da classe \textit{abntex2} no início do arquivo principal \textit{main.tex} \cite{abntex2classe}.

% Conforme a \href{https://repositorio.ufsc.br/bitstream/handle/123456789/197121/RN46.2019.pdf?sequence=1&isAllowed=y}{Resolução NORMATIVA nº 46/2019/CPG} as dissertações e teses não serão mais entregues em formato impresso na Biblioteca Universitária. Consulte o Repositório Institucional da UFSC ou sua Secretaria de Pós Graduação sobre os procedimentos para a entrega. 

% \nocite{NBR6023:2002}
% \nocite{NBR6027:2012}
% \nocite{NBR6028:2003}
% \nocite{NBR10520:2002}

% % ----------------------------------------------------------
% \section{Objetivos}
% % ----------------------------------------------------------

% Nas seções abaixo estão descritos o objetivo geral e os objetivos 
% específicos.

% % ----------------------------------------------------------
% \subsection{Objetivo Geral}
% % ----------------------------------------------------------

% Descrição...

% % ----------------------------------------------------------
% \subsection{Objetivos Específicos}
% % ----------------------------------------------------------

% Descrição...