% ----------------------------------------------------------
\chapter{Introduction}
% ----------------------------------------------------------

% Context
According to the last report \textit{Justiça em Números}, published annually by the \gls{CNJ}, by the end of 2019, there were about 77.1 million ongoing processes waiting for a solution in the Brazilian Judiciary. In total, in 2019, 30.2 million lawsuits were filled in all Judiciary, an increase of 6.8\% in relation to 2018. From those processes, about 5.2 million were filled in the \gls{JEC}~\cite{CNJ2020}. 


Justice institutions around the world are being impacted by the application of automation solutions in the most varied contexts. It is a trend the use of \gls{AI} to define a series of strategic measures both for the execution of the core activity and for strategic decisions from the point of view of management and processing flow \cite{Hartmann2019}. 

In view of  the large amount of textual data generated by the the justice institutions, empirical methods to analyze that data, such as the \gls{ML} and \gls{TM}, may be of use~\cite{Medvedeva2019, Weiss2010}.  In the literature, 
\gls{ML} and \gls{TM} based techniques have been applied in the legal domain to solve a variety of legal problems, such as the information extraction from contracts~\cite{Hassan2020}, and the prediction of decisions in lower courts and superior courts~\cite{Sulea2017, Virtucio2018}. 


% What is ML?
According to \textcite{Mitchell1997}, \gls{ML} focuses on constructing computer systems that learn through experience. Thus, such systems can learn to classify texts, control robots, predict weather and others~\cite{Sebastiani2002, Kober2013, Shi2015}. \gls{ML} based systems can learn using some approaches~\cite{Schmidhuber2015, Caruana1997}, including supervised and unsupervised. In supervised \gls{ML}, the system maps a relationship between inputs and outputs based on labeled data. In unsupervised \gls{ML}, the system tries to find patterns in the data taking into consideration the similarities among the unlabeled data points~\cite{Theodoridis2009}.

\gls{TM} relates, according to \textcite{Aggarwal2013},  to the idea of discovering and analyzing patterns, such as trends and outliers, from textual data. \gls{TM} also focus on helping users to analyze and digest information towards a better decision making. Thus, the necessity of \gls{TM} and \gls{ML} based applications emerges with the large amount of textual data  generated by users, companies, universities and so on, and human limitations to analyze it~\cite{Lecun2015, Khan2014}. 

% What is Classification, clustering and regression in this work?
\gls{TM} and \gls{ML} have been used to address challenges regarding supervised learning, such as text classification,  text regression, and unsupervised learning like text clustering~\cite{Aggarwal2013, Trusov2016, Medvedeva2019, Zhang2019}. 



% ----------------------------------------------------------
\section{Problem Definition} % Problem and Hypothesis
% ----------------------------------------------------------


% Mas os casos são decididos manualmente pelo juiz, o que pode gerar delay.

% Alta judicialização
Considering the high level of litigation in Brazil, where the Judiciary is increasingly assigned  to solve the day-to-day conflicts, less disputes (or cases) are decided with agreements between the parties in the earliest phases. So they shall wait for a judgment, that is, the judge's final  decision on the case~\cite{DictJudgment2021}, that may take from days to years~\cite{Cury2019, Mancuso2020}.

% Talvez se as partes tiverem alguma ideia da tendência de decisões em casos como o seu, eles podem fechar um acordo antes e reduzir o tempo de espera.

% Write about JEC
\gls{JEC} are Judiciary small agencies regulated by Law no. 9,099/1995, which seek to facilitate citizens' access to Justice through simpler and cost-free procedures. As a result, \gls{JEC} tend to approach the legal problems of ordinary people who find themselves involved in daily conflicts of small economic expression, whether in the purchases they make, in the services they hire or in the accidents they suffer~\cite{Watanabe1985}.


% Muito coisa pro juiz decidir
% Mas os casos são decididos manualmente pelo juiz, o que pode gerar delay.
In the  \gls{JEC}, the cases are decided manually by the judge, and there is no automation in that sense. %Thus, this leads to slowness and, as a consequence, an impact on the large number of processes pending solution~\cite{CNJ2020}. 
% Melhorar
Moreover, depending on the legal contexts,  the judge sets an amount of compensation for immaterial damage to be paid by a party to the other. However, there is no explicit rules to define them.  Thus, the amount of compensation for immaterial damage depends on the judge's interpretation of the case~\cite{Sadiku2020}. 

If the parties had more information on the possible outcomes, based on the judge's previous decisions, they would achieve a consensus and finish the case, without the need of waiting for a judgment. In this context, the use of \gls{ML} and \gls{TM} may help in the conciliation between the parties by predicting the possible outcomes based on the previous decisions.  


Thus, this work intends to make contributions to the state of the art of \gls{ML} and \gls{TM} applied to legal texts by investigating possible solutions for the prediction of legal cases outcomes.
To get such achievement, three challenges are addressed. First, the legal text, its  complexity, and vocabulary must be represented numerically allowing the use of \gls{ML} techniques. Second, the prediction of the chances of fully or partially winning or losing the case. And third, the prediction of the amount of compensation when the party wins the case. 

% Continuar aqui

% Mas há poucos trabalhos relacionados ao direito brasileiro em termos de classificação, poucos que abordam o problema de representação e nenhum abordando a regressão conforme a RSL que fizemos.

% The first challenge is addressed in several languages, like Polish, English and Chinese~\cite{Chalkidis2019, SmywiskiPohl2019}. However, from a \gls{SRL} (see Appendix \ref{ap:rsl_representation_law}), no work explored the representations of legal texts in the Portuguese language using techniques like word embeddings, \gls{BERT} and others.

In the literature, the first challenge have been addressed in several languages, like Polish, English and Chinese~\cite{Chalkidis2019, SmywiskiPohl2019}. However, from a \gls{SRL} (see Appendix \ref{ap:rsl_representation_law}), no work explored the representations of legal texts in the Portuguese language using techniques like word embeddings, \gls{BERT} and others.

The second challenge relates to classification, a supervised \gls{ML} task which tries to learn a model to map a set of records to a set of labels~\cite{Aggarwal2013}. Based on a \gls{SRL}  (see Appendix \ref{ap:rsl_ml_law}), one can see that the classification task has been applied in many legal contexts~\cite{Chalkidis2019,Hammami2019, Hassan2020}.
Although, text classification in the legal domain is a well explored subject, the papers found in the \gls{SRL} do not compare  as many techniques for \gls{DL} and Classical \gls{ML} as this work.

The third challenge can be addressed with the regression task, an supervised \gls{ML} task which aims to train a model to map the relationship between an input $x$ to a continuous output $y$~\cite{Draper1998}. According to a \gls{SRL} (See Appendix \ref{ap:rsl_regression_law}), to the best of our knowledge, the application of text regression in the legal domain is not addressed in any work in the literature. Thus, there is the possibility of many contributions in this regard.



\section{Research Question}

``Is it possible to predict the final judgment of a legal case based on its content and predict the amount of compensation for immaterial damage using \gls{ML} and \gls{TM} techniques?''

The question can be divided into three:

\begin{itemize}[noitemsep]
    %v1:\item Which representation techniques can translate the complexity of the legal language to a numerical representation to achieve legal acceptable results in \gls{ML} tasks?
    \item Does the size and specificity  of the corpus used for word embeddings training impact the performance of a text classification using such representations? 
    
    % Em caso de mudança checar também Seção 4.3
    \item Can \gls{DL} techniques outperform Classical \gls{ML} techniques on the prediction of a judgment from the \gls{JEC}?
    % v1: \item Which machine learning techniques for classification can bring an legally acceptable accuracy to the predicted lawsuit result?
    
    \item To what extent the prediction of compensation values can be \emph{accurate} and \emph{helpful} in the legal environment using regression?
    %v1: \item Which machine learning techniques for regression can bring an legally acceptable error to the predicted amount of compensation for immaterial damage?
\end{itemize}

% ----------------------------------------------------------
\section{Objectives}
% ----------------------------------------------------------

In this section, we introduce to the reader the main objective and the specific objectives necessary to achieve it.

% ----------------------------------------------------------
\subsection{Main Objective}
% ----------------------------------------------------------

To evaluate whether we can predict the final result of a legal judgment  and predict the amount of compensation for immaterial damage based on its content using \gls{ML} and \gls{TM} techniques.

% ----------------------------------------------------------
\subsection{Specific Objectives}
% ----------------------------------------------------------

In order to fulfill the main objective, the following specific objectives must be achieved:

% TODO: Check
\begin{itemize}[noitemsep]
    \item Evaluate whether the size and specificity of the corpus used for word embeddings impact on the performance a text classification using such representations.
    \item Evaluate whether \gls{DL} techniques outperform Classical \gls{ML} techniques in the prediction of judgments from \gls{JEC}.
    \item Evaluate whether the prediction of compensation values can be accurate and helpful in the legal environment using regression.
\end{itemize}

% ----------------------------------------------------------
\section{Justification}
% ----------------------------------------------------------

% Relevancia do tema.
There is an increasing interest in the literature on the applications of \gls{ML} and \gls{TM} techniques in the legal domain. From that, the concerns about the implications of \gls{AI} uses in the legal domain and others emerge~\cite{Braz2018, Davis2020}. Worldwide, such concerns provoked the debate of regulations on \gls{AI}~\cite{Cath2018}. In 2020, the \gls{CNJ} created an ordinance to regulate the uses of \gls{AI} in the Brazilian Judiciary~\cite{CNJ2020b}. Thus, the application of \gls{ML} in the legal domain has important theoretical and practical relevance.

% Como justificamos a pesquisa.
Through the construction of three \gls{SRL}, detailed in Appendix \ref{ap:rsl_ml_law}, \ref{ap:rsl_representation_law} and \ref{ap:rsl_regression_law}, we observed such increasing research interest on the application of \gls{ML} and \gls{TM} in the legal domain in the last years. With the first \gls{SRL}, we focused on the literature regarding \gls{ML} and \gls{TM} in the legal domain. As a result, most of the selected work related to many kinds of predictions in legal documents in many languages. However, until the \gls{SRL} search date, none of the papers explored and compared as many classification \gls{ML} techniques on the same dataset as proposed in this work. In general, they focus on a small set of techniques.

% Falar sobre a facilidade de se fazer experimentos com classificação e por isso a grande quantidade de trabalhos e pouco a contribuir nessa parte.

In terms of representation of legal texts, this first \gls{SRL} showed us few relevant work. However,  to address the first part of our research question, more  search in the literature  was needed. To serve as a complement, the second \gls{SRL} focus on the representation of legal texts, and the results showed us that many representation techniques have been explored in the legal domain, including word embeddings and \gls{BERT}. However, the available work involving  the Portuguese language did not focus on the aspects of representations, such as the size, the specificity of the corpus used to train them.  They just apply existing  representations pre-trained on general texts, until the \gls{SRL} search date. Thus, we further analyze how the representations pre-trained on Portuguese improve the performance in text classification when compared to those pre-trained in texts from several contexts.

The third \gls{SRL} focused on the application of regression in legal texts. However, there were no published papers in the literature on that matter, despite the use of many search keywords patterns. A broader literature search, detailed in Appendix \ref{ap:rsl_regression_law}, showed work regarding other contexts, including Financial and Health. Thus, this work's main contributions relates to the use of regression in legal texts, with the focus on predicting compensations.

In terms of theoretical relevance, this work brings  contributions in terms of representation of legal texts as it trains word embeddings representations in Brazilian legal documents to be applied in supervised and unsupervised learning tasks. In terms of predicting the judgment through classification, the  amount of published work limits our contributions to the evaluation on how the Classical \gls{ML} and \gls{DL} techniques behave when applied to judgments from \gls{JEC}. In terms of predicting the amount of compensation through regression, this work brings contributions on impact of several \gls{TM} and \gls{ML} techniques in the pipeline as the performance of the \gls{ML} models in such task.

In terms of practical contributions, the trained models and pipelines from this work can be adapted to be applied in real legal conciliation hearings in the \gls{JEC} at \gls{UFSC}. A legal expert would present and explain the predictions to the parties. Thus, we expect to help them on reaching an agreement without the need of waiting for a judgment. In this way, the litigation in the \gls{JEC} would decrease, contributing to faster and more efficient access for citizens to justice.

% ----------------------------------------------------------
\section{Research Method and Resources}
% ----------------------------------------------------------

% O método é um conjunto de passos necessários para demonstrar que o objetivo proposto foi atingido. Se os passos do método forem seguidos, os resultados devem ser convincentes
% 

To answer the main question and its three parts, the researcher followed a set of steps. Each part shares common steps.

As the first step, there is the construction of three \gls{SRL} to evaluate the State of the Art of the application of \gls{ML} and \gls{TM} in the Legal Domain, regarding representation, classification and regression. Based on this step, the research question and objectives were set.

The second step relates to the collection of the textual datasets required to answer each question. As detailed in Chapter \ref{cap:proposal}, such datasets include an unlabeled dataset of collegiate judgments from \gls{STF}, \gls{STJ} and \gls{TJ-SC} and legal judgments from \gls{JEC} at \gls{UFSC}. They also include a set of attributes extracted by a legal expert.

The third step relates to the setup and execution of \gls{ML} experiments according to each part of the research question. That is, experiments involving the training  of word embeddings and their evaluation in text classification; experiments to train and compare \gls{DL} and Classical \gls{ML} techniques in the prediction of legal judgments from \gls{JEC}; and experiments to predict the compensation for immaterial damage.

The fourth step relates to the evaluation and discussion of the achieved  results in each experiment.

In terms of resources, this work required the use of a high performance computer from the \gls{EGOV} research group at \gls{UFSC} and a set of open source libraries for \gls{TM} and \gls{ML}.

% ----------------------------------------------------------
\section{Document Organization}
% ----------------------------------------------------------

% Describe why each chapter exists. What does it propose to do?
% To achieve the main goal

The work is structured in five chapters, beginning with this introduction.

In~\textbf{\citechapter{cap:ml_text}}, we introduce the concepts and terminology required to understand the related work and the experiments we carried out. We, thus, introduce the concepts of \gls{ML} applied to text, the \gls{TM}, the theory behind the three parts of the research question and the techniques involved.

In~\textbf{\citechapter{cap:related_works}}, we highlight the relevant work to this research, based on the construction of the three \gls{SRL}. From this chapter and the \gls{SRL}, the reader can better understand how the work is positioned in relation to the literature.

In~\textbf{\citechapter{cap:proposal}}, we describe the steps to answer each part of the research question, regarding representation, classification and regression. That is, the dataset used, the pipelines, the results and the discussion on whether the experiments answered the parts of the research question.

In~\textbf{\citechapter{cap:final_remarks}}, we present the conclusions from this research based on the answers to the parts of research questions. We also present the contributions in terms of publications and gains to the state of the art. Furthermore, there is a discussion on the limitations imposed during this research in terms of the data collection, the results' application in a conciliation hearing and others. Finally, we discuss the possible improvements on this research and the future work.


