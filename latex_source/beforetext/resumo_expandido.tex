

\setlength{\absparsep}{18pt} % ajusta o espaçamento dos parágrafos do resumo
\begin{resumo}[Resumo Expandido]
	\SingleSpacing
	
\textbf{\textsf{INTRODUÇÃO}}


De acordo com o último relatório \textit{Justiça em Números}, publicado anualmente pelo Conselho Nacional de Justiça (CNJ), ao final de 2019 havia cerca de 77,1 milhões de processos em andamento aguardando solução no Judiciário brasileiro. No total, em 2019, foram 30,2 milhões de ações ajuizadas em todo o Judiciário, um aumento de 6,8\% em relação a 2018. Dessas ações, cerca de 5,2 milhões foram ajuizadas nos Juizados Especiais Cíveis (JEC) \cite{CNJ2020}.  Para fazer face aos desafios enfrentados nos sistemas judiciários no Brasil e no mundo \cite{Sadiku2020}, é cada vez maior o interesse da literatura em aplicar técnicas baseadas em aprendizado de máquina e mineração de textos na área jurídica. Os artigos publicados buscam solucionar diversos problemas jurídicos, como extração de informações de contratos \cite{Hassan2020} e previsão de decisões em tribunais inferiores e superiores \cite{Sulea2017, Virtucio2018}. O aprendizado de máquina se concentra na construção de sistemas de computador que aprendem por meio da experiência \cite{Mitchell1997}, enquanto a mineração de textos está relacionada à ideia de descobrir e analisar padrões, como tendências e outliers, a partir de dados textuais. Mineração de textos também se concentra em ajudar os usuários a analisar e digerir informações para uma melhor tomada de decisão \cite{Aggarwal2013}. 
Considerando o alto nível de judicialização no Brasil, menos casos são decididos por meio de acordos entre as partes e, por isso, essas têm de esperar pelo julgamento de um juiz, o que pode levar de dias a anos \cite{Cury2019, Mancuso2020}.
Se as partes possuíssem mais informações sobre os possíveis resultados de seus processos, baseadas em julgamentos anteriores do juíz, elas poderiam encerram o caso ainda na conciliação. Além disso, nos JECs, as decisões ainda são manuais, sem nenhum tipo de automação. O mesmo vale para decisões envolvendo danos morais, onde não há critérios claros para definição dos valores, dependendo muito da interpretação do juiz a cerca do caso.
    Portanto, este trabalho pretende contribuir para o estado da arte em aprendizado de máquina e em mineração de textos através da investigação de soluções para predição dos possíveis resultados de processos. Tais processos são provenientes do JEC localizado na Universidade Federal de Santa Catarina e envolvem falhas em serviços de transporte aéreo. No entanto, para tal predição ocorra, três desafios são colocados: a representação numérica de textos jurídicos, necessária para a aplicação destes em algoritmos de aprendizado de máquina, levando em conta sua complexidade e vocabulário próprios. Porém, não há trabalhos explorando a representação de textos jurídicos em português a partir de técnicas como o \textit{word embeddings}. O segundo desafio se refere à predição do resultado do processo, através da tarefa de classificação, onde pretende-se treinar um modelo capaz de mapear um conjunto de dados para um conjunto de finito de rótulos. O terceiro se refere ao uso de regressão para predição do valor do dano moral a partir do texto do processo. No entanto, a partir de uma revisão da literatura, pode se afirmar que regressão ainda não foram aplicadas a textos jurídicos.

\textbf{\textsf{OBJETIVOS}}

Com base na problemática de predição do resultado dos processos do JEC na UFSC, o objetivo do trabalho é avaliar se é possível prever o julgamento final do caso a partir do seu texto e prever o valor de indenização por danos morais usando técnicas de aprendizado de máquina e mineração de textos. E, para levantar possíveis soluções para os três desafios, três objetivos específicos precisam ser executados. O primeiro deles se refere à representação, onde busca avaliar se o tamanho e a especifidade dos \textit{corpora} usados para treinar modelos de \textit{word embeddings} impacta na performance na tarefa classificação de textos. O segundo a partir dos resultados do primeiro busca avaliar se técnicas de aprendizado profundo superam técnicas clássicas de aprendizado de máquina na predição do resultado dos julgamentos do JEC. E, por fim, o terceiro objetivo se refere à regressão, onde busca-se avaliar se a predição de indenização por danos morais pode ser precisa e útil no ambiente jurídico a partir do uso de regressão em textos.


\textbf{\textsf{MÉTODO DE PESQUISA}}


Para cumprir os objetivos, o pesquisador seguiu um conjunto de passos aos quais são similares para cada objetivo específico, com pequenas distinções.
Anteriormente à definição do problema, pergunta de pesquisa e objetivo de cada uma das três partes, construiu-se três Revisões Sistemáticas da Literatura (RSL) para identificar o estado arte em relação a aplicação de aprendizado de máquina e mineração de textos no direito e o uso de técnicas de representação e regressão no direito.
O passo seguinte consistiu na coleta das bases textuais necessárias para cumprir os objetivos. Tais conjuntos de dados incluiram bases textuais de acórdãos de tribunais superiores bem como julgamento do JEC. Por fim, com o ajuda da especialista em direito, foram extraídos atributos a cerca dos processos.
O terceito passo se refere à construção de experimentos de aprendizado de máquina e mineração de texto para cada objetivo.
O quarto passos se refere à avaliação dos resultados obtidos e discussão.


\textbf{\textsf{RESULTADOS E DISCUSSÃO}}

Os três conjuntos de experimentos realizados buscaram atingir os três objetivos propostos. Em relação à representação, discutiu-se o impacto do tamanho e especificade dos \textit{corpora} usado para treinar \textit{word embeddings} na performance na classificação de textos. Percebeu-se que tanto tamanho e especificidade importam. No entanto, em termos de tamanho, percebeu-se que este impacta até um certo ponto de tal forma que adicionar mais texto não traz impactos significativos. Em relação à classificação, buscou-se comparar como técnicas de aprendizado profundo e clássicas se desempenhavam na predição dos resultados dos processos do JEC e se a primeira traria resultados superiores em relação à segunda. Percebeu-se que técnicas de aprendizado clássico se saem melhor quando o texto do julgamento não apresenta a seção de dispositivo, enquanto técnicas de aprendizado profundo performaram melhor quanto tal seção estava presente. Portanto, a aplicação real da predição no JEC usaria técnicas clássicas. Em relação aos experimentos com  regressão, percebeu-se que, a partir de aprimoramentos na pipeline para predição do valor de indenização, pode-se obter resultados com qualidade aceitável para o ambiente jurídico, uma vez que o erro médio absoluto atingido foi de menos de mil reais. Portanto, com base nos resultados alcançados, é pode se afirmar que a predição do resultado e da indenização por danos morais em julgamentos do JEC é possível a partir de técnicas aprendizado de máquina e de mineração de textos.

\textbf{\textsf{CONSIDERAÇÕES FINAIS}}

A cada ano, o número de processos judiciais aguardando decisão final tende a aumentar. Portanto, é fundamental encontrar soluções para agilizar o Judiciário brasileiro em suas diversas instâncias, desde os Juizados Especiais Cíveis até o Supremo Tribunal Federal.
Neste trabalho, é proposta a aplicação das técnicas de mineração de textos e aprendizado de máquina em julgamentos do Juizado Especial Civel da UFSC para prever os possíveis resultados bem como o valor da indenização por danos morais. Assim, esta pesquisa visa contribuir para o aumento de acordos em audiências de conciliação nos juizados especiais. 
O objetivo e a pergunta de pesquisa se refereriram à possibilidade de se prever o resultado dos julgamentos do JEC a partir do seu conteúdo, bem como  o valor de indenização por danos morais. Para tanto, três desafios foram colocados para que se atingisse o objetivo geral, sendo eles a representação de textos jurídicos, a predição do resultado do julgamento, a partir de classificação e a predição da indenização por dano moral, a partir de regressão.
Em relação à representação discutiu-se o impacto do tamanho e especificidade dos \textit{corpora} para treinamento de \textit{word embeddings} na performance classificação e percebeu-se que havia influência de ambos, no entanto o tamanho apenas até certo ponto. Em relação à classificação,  percebeu-se que a predição de julgamentos do JEC foi melhor executada por técnicas clássicas de aprendizado de máquina em relação ao aprendizado profundo. Já em relação à predição de indenização por danos morais atingiu resultados aceitáveis dentro do ambiente jurídico a partir de aprimoramentos como N-Grams e atributos extraídos pela especialista em direito. Em relação ao objetivo geral, pode-se concluir que é possível prever o resultado do julgamento a partir do texto, uma vez que a melhor acurácia foi de 82,2\%. Também é possível atingir resultados acuráveis na predição de indenização por danos morais a partir de regressão com o uso dos aprimoramento sugeridos.
Como trabalhos futuros, almeja-se verificar como técnicas de representação mais recentes influenciam nos resultados de classificação e regressão. Além disso, pretende-se verificar se é possível prever o resultado e o valor de indenização a partir do uso de textos de petições iniciais. Por fim, pretende-se aprimorar os modelos criados para que as predições sejam \textit{explicáveis} e incluir mecanismos para mitigar de eventuais vieses e preconceitos que possam estar presentes na base de dados.


\textbf{\textsf{Palavras-chave}}: Aprendizado de Máquina. Classificação de Texto. Representação de Texto. Regressão de Texto. Juizado Especial Cível. Decisões Judiciais.
\end{resumo}