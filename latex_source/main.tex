% ------------------------------------------------------------------------
% ------------------------------------------------------------------------
% Modelo UFSC para Trabalhos Academicos (tese de doutorado, dissertação de
% mestrado) utilizando a classe abntex2
%
% Autor: Alisson Lopes Furlani
% 	Modificações:
%	- 27/08/2019: Alisson L. Furlani, add 'glossaries' package
%   - 30/10/2019: Alisson L. Furlani, adjusted some spacing errors and changed math fonts
%   - 17/01/2020: Alisson L. Furlani, updated certification page
%   - 07/02/2020: Alisson L. Furlani, fixed table counter bug
%   - 11/03/2020: Alisson L. Furlani, changed greek letters in math and fixed citation style
% ------------------------------------------------------------------------
% ------------------------------------------------------------------------


\documentclass[
	% -- opções da classe memoir --
	12pt,				% tamanho da fonte
	openright,			% capítulos começam em pág ímpar (insere página vazia caso preciso)
	twoside,			% para impressão no anverso. Oposto a twoside
	a4paper,			% tamanho do papel. 
	% -- opções da classe abntex2 --
	chapter=TITLE,		% títulos de capítulos convertidos em letras maiúsculas
	section=TITLE,		% títulos de seções convertidos em letras maiúsculas
	%subsection=TITLE,	% títulos de subseções convertidos em letras maiúsculas
	%subsubsection=TITLE,% títulos de subsubseções convertidos em letras maiúsculas
	% -- opções do pacote babel --
	brazil,
	english			% idioma adicional para hifenização
	%french,				% idioma adicional para hifenização
	%spanish,			% idioma adicional para hifenização
	% brazil				% o último idioma é o principal do documento
	]{abntex2}


\usepackage{setup/ufscthesisA4-alf}
\usepackage{lipsum}

% Serif fonts
%\usepackage{mathptmx}
\usepackage{palatino}
%\usepackage[lining]{ebgaramond}
\usepackage{float}
\usepackage{multirow}
% \usepackage{fourier}
% \usepackage[T1]{fontenc}

%\usepackage{librebaskerville}
%\usepackage[T1]{fontenc}

% \usepackage{librecaslon}
% \usepackage[T1]{fontenc}

% Sans Serif fonts
\usepackage[defaultsans]{lato}
\usepackage{arydshln}
\usepackage{inconsolata}

\setlength\dashlinedash{1pt}
\setlength\dashlinegap{2pt}
%\usepackage[defaultsans]{cmbright}


\newcommand{\blue}[1]{{\color{blue}#1}}
% Text font
\renewcommand{\familydefault}{\rmdefault}
\renewcommand{\ttdefault}{lmtt} 
\newcommand{\citechapter}[1]{Chapter \ref{#1}}
\newcommand{\todo}[1]{\noindent{\color{blue}\textbf{TODO:} #1}}

\addbibresource{aftertext/references.bib} % Seus arquivos de referências

% ---
% Filtering and Mapping Bibliographies
% ---
\DeclareSourcemap{
	\maps[datatype=bibtex]{
		% remove fields that are always useless
		\map{
			\step[fieldset=abstract, null]
			\step[fieldset=pagetotal, null]
		}
		% remove URLs for types that are primarily printed
%		\map{
%			\pernottype{software}
%			\pernottype{online}
%			\pernottype{report}
%			\pernottype{techreport}
%			\pernottype{standard}
%			\pernottype{manual}
%			\pernottype{misc}
%			\step[fieldset=url, null]
%			\step[fieldset=urldate, null]
%		}
		\map{
			\pertype{inproceedings}
			% remove mostly redundant conference information
			\step[fieldset=venue, null]
			\step[fieldset=eventdate, null]
			\step[fieldset=eventtitle, null]
			% do not show ISBN for proceedings
			\step[fieldset=isbn, null]
			% Citavi bug
			\step[fieldset=volume, null]
		}
	}
}
% ---

% ---
% Informações de dados para CAPA e FOLHA DE ROSTO
% ---
% FIXME Substituir 'Nome completo do autor' pelo seu nome.
\autor{Thiago Raulino Dal Pont}
% FIXME Substituir 'Título do trabalho' pelo título da trabalho.
%\titulo{Classification of legal documents and prediction of compensation based on Text Mining and Machine Learning}
\titulo{Representation, classification and regression techniques applied to legal judgments about immaterial damage due to failures in air  transport services}
% Representation, classification and regression applied to legal judgments about immaterial damage from air transport service failures

% FIXME Substituir 'Subtítulo (se houver)' pelo subtítulo da trabalho.  
% Caso não tenha substítulo, comente a linha a seguir.
%\subtitulo{subtítulo (se houver)}
\orientador{Prof. Jomi Fred Hübner, PhD.}
% \orientador[Orientadora]{Nome da orientadora, Dra.}
\coorientador{Prof. Aires José Rover, PhD.}
\ano{2021}
% FIXME Substituir '[dia] de [mês] de [ano]' pela data em que ocorreu sua defesa.
\data{28 de agosto de 2021}
% FIXME Substituir 'Local' pela cidade em que ocorreu sua defesa.
\local{Florianópolis}
\instituicaosigla{UFSC}
\instituicao{Universidade Federal de Santa Catarina}
\tipotrabalho{Dissertação}
\formacao{mestre em Engenharia de Automação e Sistemas}
\nivel{mestrado}
\programa{Programa de Pós-Gra\-du\-a\-ção em Engenharia de Automação e Sistemas}
\centro{Centro Tecnológico e Científico}
\preambulo
{%
\imprimirtipotrabalho~submetida ao \imprimirprograma~da \imprimirinstituicao para a obtenção do título de \imprimirformacao.
}
% ---

% ---
% Configurações de aparência do PDF final
% ---
% alterando o aspecto da cor azul
\definecolor{blue}{RGB}{41,5,195}
%\usepackage[a-1b]{pdfx}
% informações do PDF
\makeatletter
\hypersetup{
     	%pagebackref=true,
		pdftitle={\@title}, 
		pdfauthor={Thiago Raulino Dal Pont},
    	pdfsubject={\imprimirpreambulo},
	    pdfcreator={LaTeX with abnTeX2},
		pdfkeywords={UFSC, DAS, Machine Learning, Text Classification, Text Representation, Text Regression, Special Civel Court, Legal Judgments}, 
		colorlinks=true,       		% false: boxed links; true: colored links
    	linkcolor=black,%blue,          	% color of internal links
    	citecolor=black,%blue,        		% color of links to bibliography
    	filecolor=black,%magenta,      		% color of file links
		urlcolor=black,%blue,
		bookmarksdepth=4
}
\makeatother
% ---

% ---
% compila a lista de abreviaturas e siglas e a lista de símbolos
% ---

% Declaração das siglas
\siglalista{AI}{Artificial Intelligence}
\siglalista{ML}{Machine Learning}
\siglalista{BOW}{Bag of Words}
\siglalista{VSM}{Vector Space Model}
\siglalista{JEC}{Special Civel Court}
\siglalista{TM}{Text Mining}
\siglalista{TF}{Term Frequency}
\siglalista{TF-IDF}{Term Frequency-Inverse Document Frequency}
\siglalista{IDF}{Inverse Document Frequency}
\siglalista{DL}{Deep Learning}
\siglalista{JECs}{Special Civel Courts}
\siglalista{SRL}{Systematic Review of the Literature}
\siglalista{LDA}{Latent Dirichlet Allocation}
\siglalista{CNJ}{National Council of Justice}
\siglalista{GPT-3}{Generative Pre-trained Transformer-3}
\siglalista{NLP}{Natural Language Processing}
\siglalista{MI}{Mutual Information}
\siglalista{RFE}{Recursive Feature Elimination}
\siglalista{PCA}{Principal Component Analysis}
\siglalista{PC}{Principal Component}
\siglalista{RBF}{Radial Basis Function}
\siglalista{NMF}{Non-negative Matrix Factorization}
\siglalista{CNN}{Convolutional Neural Network}
\siglalista{LSTM}{Long-Short Term Memory}
\siglalista{Bi-LSTM}{Bidirectional Long-Short Term Memory}
\siglalista{CBOW}{Continuous Bag of Words}
\siglalista{BERT}{Bidirectional Encoder Representations from Transformers}
\siglalista{ELMo}{Embeddings from Language Model}
\siglalista{AB}{AdaBoost}
\siglalista{XGB}{XGBoosting}
\siglalista{RF}{Random Forest}
\siglalista{LR}{Logistic Regression}
\siglalista{kNN}{k Nearest Neighbors}
\siglalista{SVM}{Support Vector Machine}
\siglalista{NN}{feed-forward Neural Networks}
\siglalista{GB}{Gradient Boosting}
\siglalista{BG}{Bagging}
\siglalista{RNN}{Recurrent Neural Network}
\siglalista{DT}{Decision Tree}
\siglalista{NB}{Na\"ive Bayes}
\siglalista{TP}{True Positive}
\siglalista{TN}{True Negative}
\siglalista{UFSC}{Federal University of Santa Catarina}
\siglalista{XAI}{Explainable Artificial Intelligence}
\siglalista{FN}{False Negative}
\siglalista{FP}{False Positive}
\siglalista{t-SNE}{t-distributed Stochastic Neighbor Embedding}
\siglalista{RMSE}{Root Mean Square Error}
\siglalista{MAE}{Mean Absolute Error}
\siglalista{$R^2$}{Coefficient of Determination}
\siglalista{STF}{Federal Supreme Court}
\siglalista{STJ}{Superior Court of Justice}
\siglalista{TJ-SC}{State Court of Santa Catarina}
\siglalista{EGOV}{E-government, Digital Inclusion and Knowledge Society}
\siglalista{RG}{Ridge}
\siglalista{LOO}{Leave-One-Out}
\siglalista{LSA}{Latent Semantic Analysis}
\siglalista{SOTA}{State of the Art}
\siglalista{POS}{Part-of-Speech}
\siglalista{eJRM}{Justice Relationship Management}
\siglalista{GR}{General Repercussion}
\siglalista{UnB}{University of Brasilia}
\siglalista{NLTK}{Natural Language Toolkit}
\siglalista{GPU}{Graphics Processing Unit}
\siglalista{TJ}{State Court}
\siglalista{RTF}{Rich Text Format}
\siglalista{OCR}{Optical Character Recognition}
\siglalista{PDF}{Portable Document Format}
\siglalista{SG}{Skipgram}
\siglalista{AELE}{Attributes Extracted by the Legal Expert}
\siglalista{EV}{Ensemble Voting}
\siglalista{EN}{Elastic Net}

% Declaração dos simbolos
\simbololista{C}{\ensuremath{C}}{Circunferência de um círculo}
\simbololista{pi}{\ensuremath{\pi}}{Número pi} 
\simbololista{r}{\ensuremath{r}}{Raio de um círculo}
\simbololista{A}{\ensuremath{A}}{Área de um círculo}

% compila a lista de abreviaturas e siglas e a lista de símbolos
\makenoidxglossaries 
%\makeglossaries
% ---

% ---
% compila o indice
% ---
\makeindex
% ---



% ----
% Início do documento
% ----
\begin{document}

% Seleciona o idioma do documento (conforme pacotes do babel)
\renewcommand{\arraystretch}{1.3}
\selectlanguage{english}
%\selectlanguage{brazil}

% Retira espaço extra obsoleto entre as frases.
\frenchspacing 

% Espaçamento 1.5 entre linhas
\OnehalfSpacing

% Corrige justificação
%\sloppy

% ----------------------------------------------------------
% ELEMENTOS PRÉ-TEXTUAIS
% ----------------------------------------------------------
% \pretextual %a macro \pretextual é acionado automaticamente no início de \begin{document}
% ---
% Capa, folha de rosto, ficha bibliografica, errata, folha de apróvação
% Dedicatória, agradecimentos, epígrafe, resumos, listas
% ---
% ---
% Capa
% ---
\imprimircapa
% ---

% ---
% Folha de rosto
% (o * indica que haverá a ficha bibliográfica)
% ---
\imprimirfolhaderosto*
% ---

% ---
% Inserir a ficha bibliografica
% ---
% http://ficha.bu.ufsc.br/
\begin{fichacatalografica}
	\includepdf{beforetext/Ficha_Catalografica.pdf}
\end{fichacatalografica}
% ---

% ---
% Inserir folha de aprovação
% ---
\begin{folhadeaprovacao}
	\OnehalfSpacing
	\centering
	\imprimirautor\\%
	\vspace*{10pt}		
	\textbf{\imprimirtitulo}%
	\ifnotempty{\imprimirsubtitulo}{:~\imprimirsubtitulo}\\%
	%		\vspace*{31.5pt}%3\baselineskip
	\vspace*{\baselineskip}
	%\begin{minipage}{\textwidth}
	% O presente trabalho em nível de \imprimirnivel~foi avaliado e aprovado por banca examinadora composta pelos seguintes membros
	The present work at \imprimirnivel~level was evaluated and approved by an examining board composed of the following members:\\
	%\end{minipage}%
	\vspace*{\baselineskip}
	Prof.(a) xxxx, Dr(a).\\
	Instituição xxxx\\
	\vspace*{\baselineskip}
	Prof.(a) xxxx, Dr(a).\\
	Instituição xxxx\\
	\vspace*{\baselineskip}
	Prof.(a) xxxx, Dr(a).\\
	Instituição xxxx\\
	\vspace*{2\baselineskip}
	\begin{minipage}{\textwidth}
		Certificamos que esta é a \textbf{versão original e final} do trabalho de conclusão que foi julgado adequado para obtenção do título de \imprimirformacao.\\
	\end{minipage}
	%    \vspace{-0.7cm}
	\vspace*{\fill}
	\assinatura{\OnehalfSpacing Coordenação do Programa de Pós-Graduação}
	\vspace*{\fill}
	\assinatura{\OnehalfSpacing\imprimirorientador \\ \imprimirorientadorRotulo}
	%	\ifnotempty{\imprimircoorientador}{
	%	\assinatura{\imprimircoorientador \\ \imprimircoorientadorRotulo \\
	%		\imprimirinstituicao~--~\imprimirinstituicaosigla}
	%	}
	% \newpage
	\vspace*{\fill}
	\imprimirlocal, \imprimirano.
\end{folhadeaprovacao}
% ---

% ---
% Dedicatória
% ---
\begin{dedicatoria}
	\vspace*{\fill}
	\noindent
	\begin{adjustwidth*}{}{5.5cm} 
		\raggedleft       
		Este trabalho é dedicado aos meus colegas de classe e aos meus queridos pais.
	\end{adjustwidth*}
\end{dedicatoria}
% ---

% ---
% Agradecimentos
% ---
\begin{agradecimentos}
	Inserir os agradecimentos aos colaboradores à execução do trabalho. 
	
	Xxxxxxxxxxxxxxxxxxxxxxxxxxxxxxxxxxxxxxxxxxxxxxxxxxxxxxxxxxxxxxxxxxxxxx. 
\end{agradecimentos}
% ---

% ---
% Epígrafe
% ---
\begin{epigrafe}
	\vspace*{\fill}
	\begin{flushright}
		\textit{``Texto da Epígrafe.\\
			Citação relativa ao tema do trabalho.\\
			É opcional. A epígrafe pode também aparecer\\
			na abertura de cada seção ou capítulo.\\
			Deve ser elaborada de acordo com a NBR 10520.''\\
			(SOBRENOME do autor da epígrafe, ano)}
	\end{flushright}
\end{epigrafe}
% ---

% ---
% RESUMOS
% ---


% resumo em inglês
\begin{resumo}[Abstract]
	\SingleSpacing
	\begin{otherlanguage*}{english}
		Resumo traduzido para outros idiomas, neste caso, inglês. Segue o formato do resumo feito na língua vernácula. As palavras-chave traduzidas, versão em língua estrangeira, são colocadas abaixo do texto precedidas pela expressão “Keywords”, separadas por ponto.
		   
		\vspace{1.5em}
		\textbf{Keywords}: Keyword 1. Keyword 2. Keyword 3.
	\end{otherlanguage*}
\end{resumo}

% resumo em português
\setlength{\absparsep}{18pt} % ajusta o espaçamento dos parágrafos do resumo
\begin{resumo}[Resumo]
	\SingleSpacing
	No resumo são ressaltados o objetivo da pesquisa, o método utilizado, as discussões e os resultados com destaque apenas para os pontos principais. O resumo deve ser significativo, composto de uma sequência de frases concisas, afirmativas, e não de uma enumeração de tópicos. Não deve conter citações. Deve usar o verbo na voz ativa e na terceira pessoa do singular. O texto do resumo deve ser digitado, em um único bloco, sem espaço de parágrafo. O espaçamento entre linhas é simples e o tamanho da fonte é 12. Abaixo do resumo, informar as palavras-chave (palavras ou expressões significativas retiradas do texto) ou, termos retirados de thesaurus da área. Deve conter de 150 a 500 palavras. O resumo é elaborado de acordo com a NBR 6028.
	
	\textbf{Palavras-chave}: Palavra-chave 1.  Palavra-chave 2. Palavra-chave 3.
\end{resumo}


%% resumo em francês 
%\begin{resumo}[Résumé]
% \begin{otherlanguage*}{french}
%    Il s'agit d'un résumé en français.
% 
%   \textbf{Mots-clés}: latex. abntex. publication de textes.
% \end{otherlanguage*}
%\end{resumo}
%
%% resumo em espanhol
%\begin{resumo}[Resumen]
% \begin{otherlanguage*}{spanish}
%   Este es el resumen en español.
%  
%   \textbf{Palabras clave}: latex. abntex. publicación de textos.
% \end{otherlanguage*}
%\end{resumo}
%% ---

{%hidelinks
	\hypersetup{hidelinks}
	% ---
	% inserir lista de ilustrações
	% ---
	\pdfbookmark[0]{\listfigurename}{lof}
	\listoffigures*
	\cleardoublepage
	% ---
	
	% ---
	% inserir lista de quadros
	% ---
	\pdfbookmark[0]{\listofquadrosname}{loq}
	\listofquadros*
	\cleardoublepage
	% ---
	
	% ---
	% inserir lista de tabelas
	% ---
	\pdfbookmark[0]{\listtablename}{lot}
	\listoftables*
	\cleardoublepage
	% ---
	
	% ---
	% inserir lista de abreviaturas e siglas (devem ser declarados no preambulo)
	% ---
	\imprimirlistadesiglas
	% ---
	
	% ---
	% inserir lista de símbolos (devem ser declarados no preambulo)
	% ---
	\imprimirlistadesimbolos
	% ---
	
	% ---
	% inserir o sumario
	% ---
	\pdfbookmark[0]{\contentsname}{toc}
	\tableofcontents*
	\cleardoublepage
	
}%hidelinks
% ---
% ---

% ----------------------------------------------------------
% ELEMENTOS TEXTUAIS
% ----------------------------------------------------------
\textual


% ----------------------------------------------------------
\chapter{Introduction}
% ----------------------------------------------------------

% Here we write the context (for text mining and legal areas)
According to the last report \textit{Justiça em Números}, published annually by the National Council of Justice (CNJ), by the end of 2019, there was around 77,1 million ongoing processes waiting for a solution in the Brazilian Judiciary. In total, in 2019, 30,2 million lawsuits were filled in all Judiciary, an increase of 6,8\% in relation to 2018. From those processes, about 5,2 million were filled in the Special Civel Courts (JECs)  \cite{CNJ2020}. 

% Write about JECs
JECS are Judiciary bodies regulated by Law no. 9,099/1995, which seek to facilitate citizens' access to Justice through simpler and cost-free procedures. As a result, JECs tend to approach the legal problems of ordinary people who find themselves involved in daily conflicts of small economic expression, whether in the purchases they make, in the services they hire or in the accidents they suffer \cite{Watanabe1985}.

% The importance of the JECS and the problem of 
The judicial lawsuits processed at the JECs are decided manually by the judge, and there is no automation in that sense. This leads to slowness and, as a consequence, the large number of processes pending solution (CITE). In addition, these processes are composed of unstructured textual data that, in addition to being represented in natural language, have their own legal vocabulary (CITE). 

% Write about NLP relevance and uses (falar da RSL que foi feita e que vamos detalhar depois)

In terms of NLP techniques, there are being used to solve several tasks, such as classification (CITE), text summarization (CITE) and  regression (CITE).

\todo{Continue...}

% ----------------------------------------------------------
\section{Problem Definition} % Problem and Hypothesis
% ----------------------------------------------------------
% 

``Is it possible to predict the result of a legal case based on its content and predict the amount of compensation for immaterial damage using machine learning and text mining techniques?''

The question can be broke down into three:

\begin{itemize}[noitemsep]
    \item How to translate the complexity of the legal language to a numerical representation?
    \item Which machine learning techniques for classification can bring an legally acceptable accuracy to the predicted lawsuit result?
    \item Which machine learning techniques for regression can bring an legally acceptable error to the predicted amount of compensation for immaterial damage?
\end{itemize}

In this work, we set one hypothesis for each part of our research question as listed below:

\begin{itemize}[noitemsep]
    \item Representation techniques based on Vector Space Representation (VSM), such as Word Embeddings, trained on legal corpus, and Bag of Words (BOW) using Term Frequency (TF) values, can satisfactorily represent legal texts ;
    \item Both classical and deep machine learning techniques for classification can achieve legally acceptable results to predict the lawsuit result using the listed representations
    \item Both classical and deep machine learning techniques for regression can achieve legally acceptable results to predict the compesation value using the listed representations.
\end{itemize}
% ----------------------------------------------------------
\section{Objectives}
% ----------------------------------------------------------

In this section we introduce to the reader the main objective and the specific objectives necessary to achieve it.

% ----------------------------------------------------------
\subsection{Main Objective}
% ----------------------------------------------------------

To evaluate if we can predict with a legally acceptable amount of accuracy the result of a legal case and predict the amount of compensation using machine learning and text mining techniques.

% ----------------------------------------------------------
\subsection{Specific Objectives}
% ----------------------------------------------------------

To do that we need to:

\begin{itemize}[noitemsep]
    \item Demonstrate that representing the legal cases numerically using word embeddings and BOW can achieve legally acceptable results in the classification and regression tasks.
    \item Demonstrate that it is possible to predict the lawsuit result using classical and deep machine learning techniques for classification with legally acceptable accuracy.
    \item Demonstrate that it is possible to predict the amount of compensation  using classical and deep machine learning  techniques for regression with legally acceptable error.
\end{itemize}

% ----------------------------------------------------------
\section{Justification and Subject Relevance}
% ----------------------------------------------------------
\begin{itemize}[noitemsep]
    \item RSL Results
    \item AI Regulations and initiatives
    \item Growth in interest 
\end{itemize}

% ----------------------------------------------------------
\section{Methodological Procedures}
% ----------------------------------------------------------

\begin{itemize}[noitemsep]
    \item Systematic Review
    \item Legal Dataset 
    \item Experiments using Python
    \item Evaluation
    \item Check Legal Acceptable Accuracy and Errors
    \item Resources
\end{itemize}

% Structure
% Include here the research delimitation
% The steps we have followed to answer the question


% ----------------------------------------------------------

\section{Contributions}
% ----------------------------------------------------------

\begin{itemize}[noitemsep]
    \item Pre-trained word embeddings models for Brazilian legal texts, since there was no available representation available before.
    \item Impact of adjustments in the pipeline for regression and classification.
    \item As real life application, we would help the Judiciary by helping to end the lawsuits in JEC at the conciliation hearing step.
\end{itemize}

% Contribuições mais acadêmicas
% Contribuições mais práticas
% - Help to speed up the process by ending them at the conciliation hearing.

% ----------------------------------------------------------
\section{Document Organization}
% ----------------------------------------------------------

The work is structured in X chapters, beginning with this introduction.

In \citechapter{cap:ml_text}, we introduce...

In \citechapter{cap:related_works}, ...


\begin{equation}
    y = \sum_{x=0}^{n}\cos(x)
\end{equation}








% ============================================================================================ %







% As orientações aqui apresentadas são baseadas em um conjunto de normas elaboradas pela \gls{ABNT}. Além das normas técnicas, a Biblioteca também elaborou uma série de tutoriais, guias, \textit{templates} os quais estão disponíveis em seu site, no endereço \url{http://portal.bu.ufsc.br/normalizacao/}.

% Paralelamente ao uso deste \textit{template} recomenda-se que seja utilizado o \textbf{Tutorial de Trabalhos Acadêmicos} (disponível neste link \url{https://repositorio.ufsc.br/handle/123456789/180829}) e/ou que o discente \textbf{participe das capacitações oferecidas da Biblioteca Universitária da UFSC}.

% Este \textit{template} está configurado apenas para a impressão utilizando o anverso das folhas, caso você queira imprimir usando a frente e o verso, acrescente a opção \textit{openright} e mude de \textit{oneside} para \textit{twoside} nas configurações da classe \textit{abntex2} no início do arquivo principal \textit{main.tex} \cite{abntex2classe}.

% Conforme a \href{https://repositorio.ufsc.br/bitstream/handle/123456789/197121/RN46.2019.pdf?sequence=1&isAllowed=y}{Resolução NORMATIVA nº 46/2019/CPG} as dissertações e teses não serão mais entregues em formato impresso na Biblioteca Universitária. Consulte o Repositório Institucional da UFSC ou sua Secretaria de Pós Graduação sobre os procedimentos para a entrega. 

% \nocite{NBR6023:2002}
% \nocite{NBR6027:2012}
% \nocite{NBR6028:2003}
% \nocite{NBR10520:2002}

% % ----------------------------------------------------------
% \section{Objetivos}
% % ----------------------------------------------------------

% Nas seções abaixo estão descritos o objetivo geral e os objetivos 
% específicos.

% % ----------------------------------------------------------
% \subsection{Objetivo Geral}
% % ----------------------------------------------------------

% Descrição...

% % ----------------------------------------------------------
% \subsection{Objetivos Específicos}
% % ----------------------------------------------------------

% Descrição...
% ----------------------------------------------------------
\chapter{Machine Learning for Text}\label{cap:ml_text}

\section{Machine Learning (definição)}

\section{Text Mining and Natural Language Processing}

\begin{itemize}
    \setlength\itemsep{-0.5em}
    \item Definição de Text Mining e NLP
    \item Como se enquadram no trabalho
    \item Aplicações
\end{itemize}

\section{Text Pre-Processing}

Filtering, Lowercasing, Stopwords, etc.

\section{Text Representation}

\subsection{Bag of Words}

\subsection{Word Embeddings}

\subsection{Recent techniques (n usadas; BERT, GPT-3, ...)}

\section{Classification}

\subsection{Definition}

\subsection{Techniques}

\subsection{Evaluation Methods}

% Diff Accuracy Micro and Macro. The same for F1

\section{Regression}

\subsection{Definition}

\subsection{Techniques}

\subsection{Evaluation Methods}



\section{Applications of Machine Learning in Text}

% Uma seção só, pra falar das duas técnicas.
% 

\section{Conclusions for the Chapter}
% % ----------------------------------------------------------

% ----------------------------------------------------------
\chapter{Related Work}\label{cap:related_works}
% ---------------------------------------------------------- Citar que uma parte dos trabalhos relacionados foram selecionados a partir de RSLs e mais alguns foram selecionado como compleme.

% Citar que muitos dos trabalhos vieram das RSLs
% 

This chapter presents the related work relevant to this research. Part of the them came from the SRLs and others added as complement by the researcher.

\section{Related works in Text Representation}


\section{Related works in Text Classification}


\section{Related works in Text Regression}


\section{Conclusions for the Chapter}

% Aqui fazer uma síntese colocando o trabalho em relação à literatura. O que ele faz que não existe na literatura.

% Classificação é mais explorado pelo fato de ser mais ser mais simples enxergar problemas nessa tarefa. Há então a aplicação e avaliação de ML numa aplicação específica que é o JEC e transporte aéreo.

% Já do ponto de vista de regressão, se aplica no direito onde também ninguém resolveu explorar, tem também a questão da avaliação dos passos da pipeline.

% Já em relação a representação, tem a parte criação de representação para o direito brasileiro que não havia sido explorado com mais aprofundamento.

% Tem também a questão de a classificação ser mais branda em termos de resultado, uma vez que diz um sim ou não. Já a regressão diz um resultado específico com uma margem de erro.


% ----------------------------------------------------------
\chapter{Experiments, Results and Discussion} \label{cap:proposal}
% ----------------------------------------------------------

%----------------------------------------------------------

% Describe the order of the experiments
%----------------------------------------------------------
\section{Dataset Construction} \label{sec:dataset_construction}

\subsection{Labeled Dataset from Special Civel Court}
% JEC Dataset


% Retirado da parte de classificação. Adaptar aqui
%A base de dados de treinamento contou com um total de 665 documentos de sentenças, abrangendo as quatro classes. Entretanto, algumas destas sentenças apresentaram mais de um resultado, ou seja, contêm mais de uma classe. Nesses casos, as sentenças foram replicadas para cada uma das classes indicadas em seu dispositivo, resultando em um total de 673 sentenças para o treinamento. Assim, para a classe “procedente”, foram incluídas 214 sentenças; para classe “improcedente”, 70 sentenças; para classe “parcialmente procedente”, 379 sentenças; e para a classe “extinção”, 10 sentenças.

% Cases' result = label in this work

\subsection{Unlabeled Dataset from Brazilian Higher Courts}
% STF, STJ, TJ-SC Dataset

% Warning!! Copy/Paste 
Concerning embeddings training, the first step is to obtain the collection of legal documents from the court web portals, followed by raw text extraction from these documents. To enable us to evaluate the specificity influence of these legal corpora, we divided it into two contexts: related to general legal texts and related to air transport services text.

% Warning!! Copy/Paste
We also collected texts from other general topics (not related to legal domains) that are already compiled and freely available. Having the corpora for legal and miscellaneous contexts, we applied some processing steps to remove noise from texts. To evaluate the influence of corpus size in embeddings training, we divided these three corpora into smaller pieces based on word count.

% Warning!! Copy/Paste 
To train the embeddings it is required large text corpora to be able to get good embeddings. However, in the Brazilian Portuguese language, we could not find any dataset available on the Internet containing enough legal text corpora for our purposes. Thus, we had to build our legal corpora.

% Warning!! Copy/Paste 
Our main sources of legal text are Brazilian courts platforms. We collected judgments from the webpages of Federal Supreme Court (STF), Superior Court of Justice (STJ) and State Court of Santa Catarina (TJ-SC) \cite{STF2020, STJ2020, TJSC2020}. 
We also collected judgments from the JusBrasil portal containing processes related only to failures on air transport service from all State Courts (TJ) from Brazil \cite{Jusbrasil2020}.

% Warning!! Copy/Paste 
Table \ref{tab:count_process} shows the number of processes acquired and word count for each Tribunal:

\begin{table}[htb]
\caption{Acquired process from Courts for Embeddings Training}
\label{tab:count_process}
\centering
\begin{tabular}{@{}crrrr@{}}
\toprule
\textbf{Source}      & \multicolumn{1}{c}{\textbf{\begin{tabular}[c]{@{}c@{}}Collegial \\ Judgments\end{tabular}}}   & \multicolumn{1}{c}{\textbf{\begin{tabular}[c]{@{}c@{}}Individual \\ Judgments\end{tabular}}} & \multicolumn{1}{c}{\textbf{Subtotal}} &\multicolumn{1}{c}{\textbf{\begin{tabular}[c]{@{}c@{}}Word \\ Count\end{tabular}}} \\ \midrule
STF                  & 64,779              & 118,910              & 183,689     & 294,937,185  \\\hdashline
STJ                  & 101,141              & 0                    & 101,141      & 312,687,450 \\\hdashline
TJ-SC                & 989,964              & 662,535              & 1,652,499   & 3,060,212,814  \\\hdashline
TJs (JusBrasil)           & 34,239               & 0                    & 34,239        & 78,138,337\\ \midrule
\multicolumn{1}{l}{} & \multicolumn{1}{l}{} & \textbf{TOTAL}       & 1,971,568     &  3,745,975,786\\ \bottomrule
\end{tabular}

Source: Adapted from \textcite{DalPont2020}
\end{table}

% Warning!! Copy/Paste 
After downloading all processes, most of them in PDF and Rich Text Format (RTF) formats, we extracted raw texts from these files. We did not apply Optical Character Recognition (OCR) in scanned PDF documents, due to time limits to finish the experiments, so only digital PDFs were accounted with RTF files in Table \ref{tab:count_process}. 

% Warning!! Copy/Paste 
With the extracted texts, we applied some pre-processing steps, as discussed further in this section. 

% Warning!! Copy/Paste 
Then we built the legal text corpora containing all the processes related to all law subjects, which we call \emph{general} legal text corpora in this work. Using this base, we created another text corpora whose processes are related only to air transport and consumer law, and we call it \emph{air transport} legal text corpora.

% Warning!! Copy/Paste 
To be able to compare how good embeddings trained with legal texts perform against those created with all kinds of texts, we also created other corpora from a variety of sources. Thus, we searched for free available textual datasets. In this work, we call these texts as \emph{global} context texts. Table \ref{tab:global_corpora} shows all the global text datasets used. Then we apply some preprocessing steps, as will be described further in this section.

\begin{table}[htb]
\caption{Global context corpora}
\label{tab:global_corpora}
\centering
\begin{tabular}{@{}crrc@{}}
\toprule
\textbf{Dataset}                   & \textbf{Documents} & \textbf{Word Count} & \textbf{Source} \\ \midrule
Wikipedia in Portuguese            & 1,014,713          & 303,622,360         & \textcite{Wikipedia2019}                \\\hdashline
Brazilian Literature Books         & 169                & 37,848,783          & \textcite{Tatman2017}                 \\\hdashline
Old Newspapers                     & 617,627            & 26,441,581          &         \textcite{Tan2020}     \\\hdashline
Folha de São Paulo News            & 165,641            & 74,594,367          &                     \textcite{Marlessonn2019}             \\\hdashline
HC News Corpus                     & 494,128            & 27,170,063          &        \textcite{Christensen2016}         \\\hdashline
Blogspot Posts                     & 2,181,073          & 696,657,915         &          \textcite{Santos2018}       \\\hdashline
Wikihow Instructions               & 786,283            & 22,471,312          &        \textcite{Chocron2018}         \\ \midrule
\multicolumn{1}{r}{\textbf{TOTAL}} & 5,259,634          & 1,188,806,381       & \textbf{}       \\ \bottomrule
\end{tabular}

Source: Adapted from \textcite{DalPont2020}
\end{table}

\section{Text Classification in Legal Judgments}\label{sec:text_classification}

% Explain the overall idea of the experiments

% How did we get to this setup?

% É possível usar TM techniques para classificar corretamente as sentenças?

This section presents the results for the classification experiments involving \gls{JEC}'s legal judgments. The section starts from the clarification of the experiments' purpose, that is, how they help on answering the research question. Then, it presents the details on the datasets used for the experiments, the pipelines of the experiments for Classical \gls{ML} and \gls{DL}, and the results and discussion.

\subsection{Experiment's Purpose}

% \begin{itemize}[noitemsep]
%     \item Primeiro experimento
%     \item Sentir dificuldades
%     \item Tornar compreensível a pipeline para o especialista
%     \item orange 3 - técnicas classicas
%     \item Como ele ajuda a responder  à segunda parte pergunta?
% \end{itemize}


The experiments' purpose is to answer the second part of the research question, that is, whether \gls{DL} techniques  can achieve better performance on \gls{JEC}'s cases classification when compared to Classical \gls{ML} techniques.
% Detail here.
Through the application of Classical and \gls{DL} techniques to the prediction of the \gls{JEC} legal judgments, we try to estimate the techniques' performance. 
Based on the results, we compare the techniques on how well they perform.

Besides the techniques comparison, in this section we tested the performance of the models using two datasets: legal judgments with full text and the legal judgments without the results section. Using  such strategy, one can notice whether including or not the results part, containing textual description of the label, impact in the models performance. 


In the experiments with Classical \gls{ML} techniques, we applied the open-source software Orange Data Mining (Version 3.22)~\cite{Demsar13}. Such tool aims at offering a variety of \gls{ML} and \gls{TM} techniques to the user in a simple way, without the need of any programming language. On the other hand, the experiments with \gls{DL} techniques (not available in Orange) required the use of several tools: the Python programming language, version 3.8; the Keras framework,Version 2.4.3~\cite{Chollet2015}; the Natural Language Toolkit (NLTK), version 3.5 ~\cite{Loper02} and the Scikit-Learn framework, version 0.24.1~\cite{Pedregosa2012}.\footnote{Pipeline for Orange3 and code for Keras are available at \url{https://github.com/thiagordp/text_classification_in_legal_docs}.}


Finally, the results presented in this section relate to the first contact of the researcher with the areas of study. Thus, considering the extensive amount of publications on legal text classification (detailed in Appendix~\ref{ap:rsl_ml_law}) and the researcher's expertise, the classification experiments produced small contributions, that is, the applications of Classical \gls{ML} and \gls{DL} techniques to the prediction of the results of legal judgments from \gls{JEC}.


\subsection{Datasets}

For the classification experiments, two datasets were used. Both are similar as before (JEC/UFSC legal judgments), but smaller because it was the first experiment performed. The judgments were issued between  January 2014 to May 2019.

The difference in the legal judgments from the two datasets used resides in their distinct textual structure.  In one of them, we removed the result's part, while in the other dataset,  we kept such part. Thus, considering the structure of a legal judgment described in Section~\ref{sec:embedding_dataset}, the first dataset contains legal judgments composed of two parts and less amount of text. The second dataset has three parts and more text.

Table~\ref{tab:cap4_class_distr_label} describes the quantity of examples for each label. 

\begin{table}[htb]
\centering
\caption{Label's distributions for text classification}
\label{tab:cap4_class_distr_label}
\footnotesize
\begin{tabular}{@{}lc@{}}
\toprule
\multicolumn{1}{c}{\textbf{Label}} & \textbf{Examples} \\ \midrule
Well Founded                            & 214               \\
Partly Founded                     & 379               \\
Dismissed without prejudice        & 10                \\
Not founded                        & 70                \\ \midrule
\multicolumn{1}{r}{\textbf{TOTAL}} & 673               \\ \bottomrule
\end{tabular}
\end{table}


In terms of quantitative information from the datasets, Table~\ref{tab:info_dataset_classification} presents the average count of tokens per document and the vocabulary size after each of the preprocessing steps for each dataset. As we will describe later, distinct preprocessing steps carried out for the experiments with Classical \gls{ML} and \gls{DL}. The information of tokens per document in the experiment with Classical ML was not available due to the restrictions from Orange Data Mining.


\begin{table}[htb]
\centering
\caption{Information on the datasets for classification}
\label{tab:info_dataset_classification}
\footnotesize
\begin{tabular}{@{}crrrr@{}}
\toprule
\textbf{\begin{tabular}[c]{@{}c@{}}Experiment's\\ Preprocessing\end{tabular}} & \multicolumn{1}{c}{\textbf{\begin{tabular}[c]{@{}c@{}}Tokens per \\ document\\ (w/ result)\end{tabular}}} & \multicolumn{1}{c}{\textbf{\begin{tabular}[c]{@{}c@{}}Vocabulary Size\\ (w/ result)\end{tabular}}} & \multicolumn{1}{c}{\textbf{\begin{tabular}[c]{@{}c@{}}Tokens per \\ document\\ (w/o result)\end{tabular}}} & \multicolumn{1}{c}{\textbf{\begin{tabular}[c]{@{}c@{}}Vocabulary Size\\ (w/o result)\end{tabular}}} \\ \midrule
\textbf{Classical ML}                                                         & -                                                                                                         & 12,994                                                                                                 & -                                                                                                          & 12,898                                                                                                   \\
\textbf{DL}                                                                   & 673.0                                                                                                         & 15,377                                                                                                  & 644.2                                                                                                          & 15,224                                                                                                   \\ \bottomrule
\end{tabular}
\end{table}

Similar to Section~\ref{sec:text_representation}, some of the legal judgments from the datasets presented more than one result, that is, they had more than one distinct labels. An example of this type of judgments happens when two people file a joint lawsuit against an airline, and the judge set distinct judgments for each of them. In those cases, the documents were replicated for each of the labels indicated on its results section, culminating in a total of 673 legal judgments, in each dataset, to serve as input for the experiments. Considering distinct legal judgments only, the dataset contains 665 documents.



% Cases' result = label in this work

\subsection{Pipelines}

This section describes the pipelines and experimental setup for the experiments with Classical \gls{ML} and \gls{DL} techniques.

The first set of classification experiments focused on the application of Classical \gls{ML} techniques. In such experiments, Orange 3 served as execution environment and followed the pipeline from Figure~\ref{fig:cap4_pipeline_superv_ml}. It receives two types of input: the texts from legal judgments, in a plain text format, and their labels, that is, the judgments' results. As outputs, the pipeline makes available the trained models and the test set which are passed to the prediction and evaluation step.


\begin{figure}[htb]
    \centering
    \caption{Pipeline for legal text classification using Classical ML techniques}
    \label{fig:cap4_pipeline_superv_ml}
    \includegraphics[width=\textwidth]{images/chapters/cap4_classification_pipeline.pdf}
\end{figure}


The first step in the pipeline is the data preprocessing. Considering the available techniques for preprocessing textual data, described in Chapter 2 and in the literature, in Appendix~\ref{ap:rsl_ml_law},  we applied transformation, tokenization, stemming and filtering, previously discussed in Section \ref{sec:bow}:

\begin{itemize}[noitemsep]
    \item \textbf{Tranformation:} the conversion to lower case to standardize the spelling of words.
    \item \textbf{Tokenization:} the application of regular expression ($\setminus w+$) to detect the pieces of texts, while removing spaces, symbols, and punctuation.
    \item \textbf{Stemming:} To reduce the variability of similar words, we applied Porter Stemmer~\cite{Porter1980}, a simple and efficient stemming algorithm to the Portuguese language. However, it may make errors, such as contracting the word \textit{morais} to \textit{morai}, instead of \textit{moral} (Portuguese words for singular and plural of \textit{moral}, respectively).
    \item \textbf{Filtering:} Removing stopwords, such as prepositions and articles to keep only the meaningful words.
\end{itemize}


The next step in the pipeline relates to the extraction of N-Grams, which detects sequences of two or more words that appear together consistently in the text. In this research, the limit of the length of N-grams was two. Bigger numbers of N-grams would lead to large textual representations.



After N-Grams Extraction, the numerical representations of the documents are created using the algorithm \gls{BOW}. In the experiments with Classical \gls{ML}, we used the \gls{TF} to calculate the values of the BOW model. 


The next step consists on dividing the dataset in two subsets: train and test sets. As described in Chapter~\ref{cap:ml_text}, there is the cross validation. In this research, $k$ was set to 10, that is, 90\% of the dataset used for training and 10\% for testing the models. Such proportion is a common choice to avoid bias while keeping some level of variance in the division of the folds~\cite{Airola2011}

The next step consists on training the models to predict the result of the judgments from \gls{JEC} according to the possible outcomes, as described in Section~\ref{sec:text_representation}. 
Experiments involved the following techniques: \gls{kNN}, \gls{SVM}, \gls{RF}, \gls{NN}, \gls{NB} and \gls{LR}. Table~\ref{tab:hyperparam_classification} presents the hyper-parameters applied to each technique, based on the values suggested by Orange 3.


% Incluir Tabela para descrever os parâmetros dos modelos.
\begin{table}[htb]
\centering
\caption{Hyperparameters for classification techniques}
\label{tab:hyperparam_classification}
\footnotesize
\begin{tabular}{cl}
\toprule
\textbf{Technique} & \multicolumn{1}{c}{\textbf{Hyper-parameters}} \\ \midrule

\gls{kNN} & \begin{tabular}[c]{@{}l@{}}Number of Neighbors: 4;\\ Distance Metric: Euclidean;\\ Weight: Uniform\end{tabular} \\ \hdashline

\gls{LR} & \begin{tabular}[c]{@{}l@{}}Regularization type: Ridge (L2);\\ C (strength): 1\end{tabular} \\\hdashline

\gls{NB} & -- \\ \hdashline

\gls{NN} & \begin{tabular}[c]{@{}l@{}}Hidden Layers: 2\\ Neurons in each layer: 100, 50;\\ Activation Function: tanh;\\ Solver: Stochastic Gradient descent (SGD)\end{tabular} \\ \hdashline

\gls{RF} & \begin{tabular}[c]{@{}l@{}}Number of Trees: 10;\\ Minimum subset size: 5\end{tabular} \\ \hdashline

\gls{SVM} & \begin{tabular}[c]{@{}l@{}}C (cost): 1.0;\\ $\epsilon$ (Regression loss): 0.1;\\ Kernel: \gls{RBF};\\ Iteration Limit: 100\end{tabular} \\ \bottomrule

\end{tabular}
\end{table}


Finally, there is the evaluation step, which estimates how well the models performed on predicting unseen legal judgments. To measure the performance, we used the Accuracy, detailed in Chapter~\ref{cap:ml_text}. At the time these experiments were published, only the accuracy was used.

The second set of experiments related to the application of \gls{DL} techniques to the prediction of the legal judgments' results. The pipeline used for these experiments is presented in Figure~\ref{fig:pipeline_dl}.

\begin{figure}[htb]
    \centering
    \caption{Pipeline for legal text classification using DL techniques}
    \label{fig:pipeline_dl}
    \includegraphics[width=\textwidth]{images/chapters/cap4_proposed_pipeline_dl.pdf}
\end{figure}

Such steps in the pipeline are similar to those in Figure~\ref{fig:cap4_pipeline_superv_ml}, however there are distinct settings. 
Thus, the pipeline receives three types of input: the legal judgments' text, their labels and the pre-trained word embeddings  models. 

Following, the preprocessing step prepares the text to the next steps in the pipeline, however with  different settings from Figure~\ref{fig:cap4_pipeline_superv_ml}. There is the transformation, tokenization using regular expression (\textbackslash w+), however we did not apply stemming or filtering, following the literature~\cite{Mikolov2013, Pennington2014}. Thus, to reproduce the aspects of the corpus applied to the pre-training of the embeddings, their application in ML tasks may not include those preprocessing techniques.
After preprocessing there is the text representation, where pre-trained embeddings techniques were applied. We selected the pre-trained embeddings based on best results in Section~\ref{sec:text_representation}. That is, the word embeddings pre-trained using the corpus related to air transport only. In the referred section, only the GloVe technique was applied and tested due to time limits. However, we later trained other word embeddings in the same corpora to apply in the classification experiments, using the default parameters from Gensim~\cite{Radim2010}.

The next step, cross-validation,  follows the setup from the Classical ML experiments, i.e., the number of folds set to ten. 

Later, there is the models training step, involving three \gls{DL} techniques: \gls{CNN}, \gls{LSTM} and Bi-LSTM with Self Attention. The \gls{CNN} used has the same hyperparemeters  from Section~\ref{sec:text_representation}. 

The \gls{LSTM} architecture, as shown in Figure~\ref{fig:cap4_lstm_model}, receives as input the embeddings, which can be fine-tuned during the training process. Such a setting is enabled in this architecture. The embeddings pass to a Spatial Dropout layer, a regularization method to avoid overfitting on recurrent networks, especially when embeddings can be fine-tuned~\cite{Gal2016}. Then, there is the \gls{LSTM} layer with $100$ units with a dropout and a recurrent dropout of $0.2$. Finally, the output layer corresponds to a Dense Layer with four neurons having sigmoid as activation function.

\begin{figure}[htb]
    \centering
    \caption{LSTM architecture for text classification}
    \label{fig:cap4_lstm_model}
    \includegraphics[width=\textwidth]{images/chapters/cap4_lstm_model.pdf}
\end{figure}

% TODO: Definir no C2, LSTM, CNN, Bi-LSTM e Self Attention
The architecture for Bi-LSTM with Self Attention is shown in Figure~\ref{fig:cap4_bilstm_attention_model}. It starts with the Embedding layer followed by Spatial Dropout, both with the same settings from previous \gls{LSTM} model. The Bi-LSTM layer follows, containing 100 units. Then, there is the Many to One Attention step composed of  the Self-Attention, the Concatenation with last Bi-LSTM Hidden State and the Attention Vector. Finally, there is the Dense Layer as the output layer with four neurons with sigmoid activation function.

\begin{figure}[htb]
    \centering
    \caption{Bi-LSTM with Self-Attention architecture for text classification}
    \label{fig:cap4_bilstm_attention_model}
    \includegraphics[width=\textwidth]{images/chapters/cap4_bi-lstm_self_attention_model.pdf}
\end{figure}

% Adam with default parameters from Keras.

To run the \gls{DL} experiments based on the architecture from Figure~\ref{fig:pipeline_dl}, distinct setups have been made considering the three \gls{DL} techniques, the two datasets, and the five embeddings models. %Thus, there were thirty combinations of inputs in the experiments.


\subsection{Results and Discussion}


The results obtained after applying the sets of input to the pipelines from Figure~\ref{fig:cap4_pipeline_superv_ml} and~\ref{fig:pipeline_dl} are presented in the following paragraphs.
%
Table~\ref{tab:cap4_class_with_result} details the accuracy obtained by the techniques and representations on each label using the full judgments' text.

% Please add the following required packages to your document preamble:
% \usepackage{booktabs}
% \caption{Acurácia da classificação com o dispositivo das sentenças}
\begin{table}[htb]
\centering
\caption{Classification accuracy for the dataset with judgments' result}
\label{tab:cap4_class_with_result}
\footnotesize
\begin{tabular}{@{}cccrrrrr@{}}
\toprule
\textbf{Technique}&\multicolumn{1}{c}{\textbf{\begin{tabular}[c]{@{}c@{}}Type of\\\gls{ML}\\technique\end{tabular}}} & \textbf{Representation} & \multicolumn{1}{c}{\textbf{\begin{tabular}[c]{@{}c@{}}Well\\ founded\end{tabular}}} & \multicolumn{1}{c}{\textbf{\begin{tabular}[c]{@{}c@{}}Partly\\ founded\end{tabular}}} & \multicolumn{1}{c}{\textbf{\begin{tabular}[c]{@{}c@{}}Not\\ founded\end{tabular}}} & \multicolumn{1}{c}{\textbf{\begin{tabular}[c]{@{}c@{}}Dismissed\\ without\\ prejudice\end{tabular}}} & \multicolumn{1}{c}{\textbf{\begin{tabular}[c]{@{}c@{}}Total\\Acc\end{tabular}}} \\ \midrule
\textbf{\gls{kNN}} & Classical & BOW TF & 75,9\% & 71,9\% & 93,5\% & 95,7\% & 75,8\% \\
\textbf{\gls{LR}} & Classical  & BOW TF & 87,1\% & 86,2\% & 97,6\% & 97,9\% & 87,8\% \\
\textbf{\gls{NB}} & Classical  & BOW TF & 75,2\% & 47,3\% & 91,4\% & 19,6\% & 60,3\% \\
\textbf{\gls{NN}} & Classical  & BOW TF & 85,9\% & 84,9\% & 97,5\% & 97,9\% & 86,7\% \\
\textbf{\gls{RF}} & Classical  & BOW TF & 82,6\% & 82,6\% & 96,1\% & 97,9\% & 84,2\% \\
\textbf{\gls{SVM}} & Classical  & BOW TF & 34,3\% & 45,5\% & 89,6\% & 98,5\% & 47,3\% \\
\textbf{Bi-LSTM-SA} & \gls{DL}& FT CBOW & 81,1\% & 80,4\% & 96,6\% & 98,2\% & 78,1\% \\
\textbf{Bi-LSTM-SA} & \gls{DL} & FT \gls{SG} & 81,1\% & 80,2\% & 97,3\% & 98,5\% & 78,6\% \\
\textbf{Bi-LSTM-SA}  & \gls{DL} & GloVe & 83,1\% & 82,6\% & 96,9\% & 98,2\% & 80,4\% \\
\textbf{Bi-LSTM-SA} & \gls{DL}  & W2V CBOW & 81,3\% & 80,1\% & 97,5\% & 98,4\% & 78,6\% \\
\textbf{Bi-LSTM-SA}  & \gls{DL} & W2V \gls{SG} & 80,8\% & 79,6\% & 97,5\% & 98,4\% & 78,2\% \\
\textbf{\gls{CNN}} & \gls{DL}  & FT CBOW & 48,3\% & 48,9\% & 90,2\% & 98,5\% & 42,9\% \\
\textbf{\gls{CNN}} & \gls{DL}  & FT \gls{SG} & 94,0\% & 94,8\% & 96,9\% & 98,5\% & 92,1\% \\
\textbf{\gls{CNN}}  & \gls{DL} & GloVe & \textbf{97,2\%} & \textbf{97,5\%} & \textbf{98,1\%} & 98,4\% & \textbf{95,5\%} \\
\textbf{\gls{CNN}}  & \gls{DL} & W2V CBOW & 73,0\% & 68,7\% & 90,6\% & 98,5\% & 65,4\% \\
\textbf{\gls{CNN}} & \gls{DL}  & W2V \gls{SG} & 96,0\% & 96,9\% & 97,6\% & 98,5\% & 94,5\% \\
\textbf{\gls{LSTM}} & \gls{DL}  & FT CBOW & 78,2\% & 74,6\% & 90,5\% & 98,5\% & 70,9\% \\
\textbf{\gls{LSTM}}  & \gls{DL} & FT \gls{SG} & 80,2\% & 76,4\% & 92,9\% & 98,5\% & 74,0\% \\
\textbf{\gls{LSTM}} & \gls{DL}  & GloVe & 80,7\% & 78,6\% & 96,1\% & 98,5\% & 77,0\% \\
\textbf{\gls{LSTM}}  & \gls{DL} & W2V CBOW & 77,4\% & 73,8\% & 89,6\% & 98,5\% & 69,7\% \\
\textbf{\gls{LSTM}} & \gls{DL}  & W2V \gls{SG} & 78,6\% & 73,7\% & 91,8\% & 98,5\% & 71,3\% \\ \bottomrule
\end{tabular}

%\fonte{\textcite{Sabo2019}.}
\end{table}


% O que escrever aqui?
% Primeiro dizer o óbvio: qual a ordem da melhor pra pior?
% CNNs..., LR e tudo mais
% CNN foi muito bem, soube detectar bem quais palavras tem alguma relação com o resultado.
% E as demais técnicas, o que mudou?
% Diferença entre os resultados dos embeddings.
% Em relação às classes? continua igual?
% WF: CNN Glove, W2V SG, FT SG, 
% PF: idme
% NF:
% DWJ: 
% TOAL:
% Pq? 
% - Possíveis causas para bom resultado:
%       Facilidade em relacionar o texto descrito com o resultado, mas não uma capacidade de "raciocinar" 


From Table~\ref{tab:cap4_class_with_result}, one can notice that most techniques achieved accuracies superior to 70\%, indicating good results on the classification of legal judgments from \gls{JEC}. In terms of techniques' performances, the CNN with GloVe embeddings achieved the best accuracies for the  \emph{Well Founded}, \emph{Partially Founded},  \emph{Not founded} labels and for the total accuracy. The \gls{CNN} also achieve good results when combined with the Word2Vec \gls{SG} and FastText \gls{SG}. Besides \gls{CNN}, the next techniques with best performances are Classical \gls{ML} techniques, that is, \gls{LR}, \gls{NN} and \gls{RF}. 

In terms of worst results, the \gls{CNN} with FastText \gls{CBOW} embeddings achieved the smallest total accuracy, followed by \gls{SVM}, Naïve Bayes and \gls{CNN} with Word2Vec CBOW. Regarding labels, on the other hand,  \gls{SVM} achieved the worst results for the  \emph{Well Founded}, \emph{Partially Founded},  \emph{Not founded}, and the Naïve Bayes for the \emph{Dismissed without Prejudice} label.



The second part of experiments on Classical \gls{ML} and \gls{DL} related to the classification of legal judgments from \gls{JEC} without the judgments' result. Table~\ref{tab:cap4_class_without_result} contains the accuracy achieved by each technique and representation for the four labels and total accuracy using the text of the judgments without result. 


\begin{table}[htb]
\centering
\caption{Classification accuracy for the dataset without judgments' result}
\label{tab:cap4_class_without_result}
\footnotesize
\begin{tabular}{@{}cccrrrrr@{}}
\toprule
\textbf{Technique} &\multicolumn{1}{c}{\textbf{\begin{tabular}[c]{@{}c@{}}Type of\\\gls{ML}\\technique\end{tabular}}} & \textbf{Representation} & \multicolumn{1}{c}{\textbf{\begin{tabular}[c]{@{}c@{}}Well\\ Founded\end{tabular}}} & \multicolumn{1}{c}{\textbf{\begin{tabular}[c]{@{}c@{}}Partly\\ founded\end{tabular}}} & \multicolumn{1}{c}{\textbf{\begin{tabular}[c]{@{}c@{}}Not\\ founded\end{tabular}}} & \multicolumn{1}{c}{\textbf{\begin{tabular}[c]{@{}c@{}}Dismissed\\ without\\ prejudice\end{tabular}}} & \multicolumn{1}{c}{\textbf{\begin{tabular}[c]{@{}c@{}}Total\\Acc\end{tabular}}} \\ \midrule
\textbf{kNN} & Classical & BOW TF & 75,2\% & 72,1\% & 90,9\% & 93,2\% & 75,4\% \\
\textbf{LR} & Classical  & BOW TF & 80,7\% & 79,2\% & 94,9\% & 97,9\% & 81,6\% \\
\textbf{NB} & Classical  & BOW TF & 74,3\% & 46,5\% & 90,2\% & 18,1\% & 59,5\% \\
\textbf{NN} & Classical  & BOW TF & 81,0\% & 79,0\% &\textbf{ 95,1\%} & 97,9\% & 81,6\% \\
\textbf{RF}  & Classical & BOW TF & \textbf{82,2\%} & \textbf{79,9\%} & 92,7\% & 97,9\% & \textbf{82,2\%} \\
\textbf{\gls{SVM}} & Classical  & BOW TF & 33,4\% & 45,0\% & 89,6\% & 98,5\% & 46,7\% \\
\textbf{Bi-LSTM-SA} & \gls{DL} & FT CBOW & 79,5\% & 77,1\% & 94,3\% & 98,5\% & 74,7\% \\
\textbf{Bi-LSTM-SA}  & \gls{DL} & FT \gls{SG} & 78,5\% & 75,8\% & 94,6\% & 98,5\% & 73,7\% \\
\textbf{Bi-LSTM-SA}  & \gls{DL} & GloVe & 79,9\% & 77,3\% & 94,8\% & 98,4\% & 75,2\% \\
\textbf{Bi-LSTM-SA} & \gls{DL}  & W2V CBOW & 79,5\% & 76,7\% & 93,9\% & 98,5\% & 74,3\% \\
\textbf{Bi-LSTM-SA}  & \gls{DL} & W2V \gls{SG} & 80,4\% & 77,4\% & 93,8\% & 98,5\% & 75,0\% \\
\textbf{CNN}  & \gls{DL} & FT CBOW & 50,5\% & 51,7\% & 89,9\% & 98,5\% & 45,3\% \\
\textbf{CNN}  & \gls{DL} & FT \gls{SG} & 80,8\% & 78,0\% & 93,0\% & 98,5\% & 75,2\% \\
\textbf{CNN} & \gls{DL}  & GloVe & 80,4\% & 77,3\% & 93,0\% & 98,5\% & 74,6\% \\
\textbf{CNN}  & \gls{DL} & W2V CBOW & 66,6\% & 58,7\% & 90,0\% & 98,5\% & 56,9\% \\
\textbf{CNN} & \gls{DL}  & W2V \gls{SG} & 81,7\% & 79,5\% & 93,3\% & 98,5\% & 76,5\% \\
\textbf{LSTM}  & \gls{DL} & FT CBOW & 79,8\% & 71,3\% & 88,6\% & 98,5\% & 69,1\% \\
\textbf{LSTM}  & \gls{DL} & FT \gls{SG} & 77,5\% & 69,1\% & 88,1\% & 98,5\% & 66,6\% \\
\textbf{LSTM} & \gls{DL}  & GloVe & 81,1\% & 74,7\% & 91,2\% & 98,5\% & 72,8\% \\
\textbf{LSTM}  & \gls{DL} & W2V CBOW & 80,8\% & 73,7\% & 87,8\% & 98,5\% & 70,4\% \\
\textbf{LSTM} & \gls{DL}  & W2V \gls{SG} & 79,2\% & 72,4\% & 88,4\% & 98,5\% & 69,2\% \\ \bottomrule
\end{tabular}

%\fonte{\textcite{Sabo2019}.}
\end{table}

The results show the sequence of best techniques has changed, as the Classical ML techniques \gls{RF}, \gls{LR} and \gls{NN} achieved the best total accuracies followed by the \gls{CNN} with Word2Vec \gls{SG}. In terms of labels, the RF performed better for the \emph{Well founded} and  \emph{Partially founded}, while \gls{NN} performed better with \emph{Not founded}. For the label \textit{Dismissed without prejudice}, several classes achieved good results, although it has the smallest sample. On the other hand, similar to the results from Table~\ref{tab:cap4_class_with_result}, the techniques with worse performance were the \gls{CNN} with FastText CBOW followed by \gls{SVM} and \gls{CNN} with Word2Vec CBOW.


% Results para colocar aqui
% 
% A CNN ele se saiu muito bem com resultado e não tão bem sem  o resultado o que mostra que ela conseguiu relacionar muito bem o trecho da sentença com o label do doc, isto é. A CNN identificou a sentença como uma parte muito importante do documento.
% Além disso, percebeu-se a importância da técnicna de representação, uma que vez que simplesmente alterando a representação utilizada na CNN fez que com ela tivesse tanto resultados muito bons quanto muito ruins.
% Deixar claro que o caso prático seria usar o texto sem a sentença final, apenas contentdo o resumo do caso e as leis aplicáveis. Portanto que num caso real no JEC, as técnicas de Classical ML seriam mais adequadas para a classificação.


In general, it is inferred that the accuracy in the experiment with the removal of the judgments' result (part of the text indicating the label to which it belongs) suffered a minimal reduction for the Classical ML techniques as well as LSTM and Bi-LSTM-SA, which demonstrates that the classifiers were able to maintain their performance with the text in which the facts narrated by the parties to the process are reported and the legal grounds applicable to the case. 
However, for the CNN there was a significant decay in performance when removing the cases' result. And, such  decay in performance may show the technique inferred that the text from the result part  had significant information for the classification of the legal cases.
However, the removal of such part made it harder for the CNN to infer the judgment's label based on remaining two parts of the text. 

Another observation for the CNN is the impact of the representations in the performance of the technique, as changing the representation used in the pipeline was enough to reduce the accuracy by more than 30\% when comparing to the best CNN results. However, for the other two DL techniques, LSTM and Bi-LSTM with Self Attention, the differences in performance while shifting representations were significantly smaller.

% Esse fica como último parágrafo.
Finally, a good performance when classifying the judgments without results becomes a prerequisite for carrying out experiments with texts from legal proceedings in which there is still no sentence, that is, in which the judge has not yet decided the result. In this case, the best choice may be the use of Classical ML techniques, such as LR and RF, as they perform better on the classification task than DL techniques and due to the fact those models are less complex and require less examples to train.
The \gls{DL} techniques, especially CNN with GloVe may be useful inside JEC when the objective is to organize existing judgments by categories such as their labels or matters, for example. 

This conclusions are limited to our small dataset from \gls{JEC}. Thus, if we had a larger dataset with more training examples, the \gls{DL} would possibly achieve better results.


% Place this in Conclusions Chapter
% \subsection{Conclusions from the section}

% The experiments from this section dealt with initial experiments carried out in the sentences of the JEC/UFSC, indicating only four possible classes to which the texts belong. In general, it was possible to obtain an overview of how the several ML techniques behave in the face of legal texts (specific on Consumer Law and failure in air transport service), evaluating which classification models reached higher and lower accuracy.

%As future works, the aim is to improve the text representation of the bag of words for a more robust representation, such as Word Embeddings, capable of numerically associating contexts and semantics to words. In addition, we intend to build a new structure for sentence prediction, composed of cascading classifiers, using attributes extracted from texts through clustering. Thus, it is possible to make intermediate classifications that will serve as a basis for a final classification. 






\section{Text Representation in Legal Judgments}


\subsection{Corpus Processing}

% Warning!! Copy/Paste
% Pré-processamento
After text extraction from the documents, we applied some pre-processing steps, which are required before training the embeddings or text classification. The first of them was the conversion to lower case. Then punctuation marks were removed, as well as special characters and some symbol characters. We removed stopwords neither apply stemmization or lemmatization, following the literature  \cite{Mikolov2013, Pennington2014}.

% Warning!! Copy/Paste
% Sobre a divisão da base maior em bases de tamanho menor
In relation to our three corpora used in embeddings training, which comprising 3.7 billion, 100 million  and 1.19 bilion words for \emph{general}, \emph{air transport} and \emph{global} corpora, respectively, we created others based on them according to the following smaller corpora sizes, considering the word count: 1,000; 10,000; 50,000; 100,000; 200,000; 500,000; 1,000,000; 5,000,000; 10,000,000; 25,000,000; 100,000,000; 500,000,000; 750,000,000 and 1,000,000,000.

% Warning!! Copy/Paste
% Porque dividimos nesses tamanhos de base.
We choose these corpora sizes to be able to compare the variation on evaluation metrics while increasing corpora size. For the air transport context, we could not embrace all these sizes due to limited corpora available. The largest sub-base had  100 million words for this context. 

% Warning!! Copy/Paste
% Cada uma dessas bases vai ser usada pra treinar um embedding diferente.
Finally, each of these smaller corpora was used to train one different word embeddings representation.

\subsection{Embeddings Training}

% Warning!! Copy/Paste
% Usamos esse algoritmo pois tem dado bons resultados e também pelo tempo de treinamento ser bem menor
In this work, we chose GloVe representation due to its good results in many NLP tasks, including text classification, and also for its training time which is significantly less than other techniques like Word2Vec and FastText \cite{Pennington2014}.  
% Parâmetros do algoritmo
In terms of GloVe parameters, we kept most of the default values, except for windows size, training iterations, and vector size, which were set to 5, 100, and 100, respectively. With these values, we achieved better results in text classification.

% Warning!! Copy/Paste
Considering the corpus sizes described in Section \ref{sec:corpus_construction} and the parameters above described, we trained 15 representations for \emph{general} and \emph{global} contexts bases. For \textit{air transport} context base, we trained 11 embeddings.

\subsection{Embeddings Evaluation in Legal Text Classification}

% Warning!! Copy/Paste
To evaluate the GloVe embeddings representations, we applied each of them to the task of text classification on judgments from JEC/UFSC. Also, we used Convolutional Neural Networks as a classification model based on the literature \cite{Kim2014}. Fig. \ref{fig:cnn_model} illustrates this model.

% Demonstrar um gráfico com a CNN.
\begin{figure}[htb]
    \centering
    \includegraphics[width=\textwidth]{images/cnn_model.pdf}
    \caption{CNN Model for Text Classification \cite{Kim2014}}
    \label{fig:cnn_model}
\end{figure}

% Warning!! Copy/Paste
% A CNN mostrada leva em conta a ordem das palavras
This CNN takes into account the order of the words by stacking the corresponding embeddings for each word as they occur in the text. 
% São aplicados filtros de convolução com diversas dimensão, onde a largura corresponde ao tamanho dos embeddings e a altura o número de embeddings que serão englobados pelo filtro.
% Warning!! Copy/Paste
Them it applies multiple convolutional masks with different dimensions that correspond to the red and yellow contours in Fig. \ref{fig:cnn_model}. Mask widths are equal to word embedding size while the heights can vary. In this context, mask height can be related to the idea of N-Grams, since they embrace multiple embeddings at the same time. 
% No original tinha 3 tamanhos de filtro, mas incluímos um mais que melhorou os resultados. Em número de filtros, tinha 100 e abaixamos pra 10 sem piorar o desempenho, mas tornando
% Warning!! Copy/Paste
In the original model, these heights were set to 3, 4, and 5. We added one more mask of height 2, which increased classification metrics. Also, we set to 10 the number of masks for each of these sizes, without affecting our results, but decreasing the required training time. 

% Warning!! Copy/Paste
% Explicar sobre como esse modelo foi utilizado
In this work, we applied each of the embeddings trained in conjunction with the CNN described in the classification of JEC/UFSC judgments, where Out of Vocabulary (OOV) words are replaced by an vector of random values. Thus, we trained and tested 41 models. Furthermore, due to the stochastic nature of neural networks training methods \cite{Cohen1995}, each of these models was trained and tested 200 times and the resulting evaluation metrics were averaged. 

% Warning!! Copy/Paste
Finally, we compare the performance in classification using Accuracy and Macro F1-Score.


% \subsection{Results Analysis and Discussion}

% % Warning!! Copy/Paste
% In the following sections, we present and discuss our results for text classification using trained embeddings for \emph{global}, \emph{general}, and \emph{air transport} contexts with multiple corpus training sizes. %Then, these results are discussed from contexts and corpus size perspectives.

% Warning!! Copy/Paste
% Experimental results
\subsection{Experimental Results}

% Warning!! Copy/Paste
Following the steps presented in section~\ref{sec:experiments}, we trained all 41 word embeddings representations for GloVe. 

% Warning!! Copy/Paste
To illustrate how these embeddings behave, in Fig.~\ref{fig:projection}, we used Principal Component Analysis (PCA) to create a projection in two dimensions of a set of words from \textit{general} context embedding trained with 1 billion words.

\begin{figure}[htb]
    \centering
    \includegraphics[width=0.9\textwidth]{images/projection.pdf}
    \caption{Word Embeddings Projection}
    \label{fig:projection}
\end{figure}

% Warning!! Copy/Paste
%%%% FIGURE- Embeddings projection
Using each embedding, we trained and tested CNNs for text classification in JEC/UFSC judgments. These two steps were repeated 200 times, and the evaluation metrics were averaged for each group of repetitions.

% Warning!! Copy/Paste
In Fig. \ref{fig:accuracy_plot} and \ref{fig:f1_plot}, we present the results, for accuracy and F1-Score, respectively, from test data applied to each CNN model. These results are related to embeddings trained with \textit{general}, \textit{air transport}, \textit{global} texts. 
The x-axis denotes the corpus sizes used to train the embeddings, while the y-axis represents accuracy or F1-Score. Each data point represents the average of the evaluation metric, after 200 train and test repetitions using each specific embedding. %Finally, we delimited y-axis to smaller intervals so the metrics variations could be visible.


% JH In the next section, we discuss these results.

\begin{figure}[htb]
    \centering
    \includegraphics[width=0.9\textwidth]{images/acc_mean_final.pdf}
    \caption{Accuracy for test set from CNN model}
    \label{fig:accuracy_plot}
\end{figure}

\begin{figure}[htb]
    \centering
    \includegraphics[width=0.9\textwidth]{images/f1_score_mean_final.pdf}
    \caption{Macro F1-Score for test set from CNN model}
    \label{fig:f1_plot}
\end{figure}

% Evaluations from context perspective
\subsection{Discussion from Context Perspective}

% Warning!! Copy/Paste
In this section, we will consider the first part of our research question: Does the specificity of the corpora in embeddings training contribute to the quality of the classification? 

% Warning!! Copy/Paste
In terms of accuracy, when we compare \emph{global} against others (Fig.~\ref{fig:accuracy_plot}), we have that higher text specificity leads to better results, for most of the corpus sizes used for embeddings training. Furthermore, when comparing \textit{general} and \textit{air transport} curves, there is a significant difference in accuracy only for the lowest and highest x-values. 
However, in terms of F1-Score, as shown in Fig. \ref{fig:f1_plot}, our observations change, once \textit{general} and \textit{air transport} curves have a similar shape. Also, for the highest corpus sizes, \textit{general} and \textit{global} curves converge to similar values of F1-Score. 
%
% Warning!! Copy/Paste
We believe that these differences in accuracy and F1-Score emerge from the fact that our dataset to text classification is imbalanced, once the former does not take this fact into account, while the latter does. However, this result still requires further investigation. 
% Warning!! Copy/Paste
In general, we can note that for smaller corpora size for embeddings training, text specificity has a more impact than for large sizes.


% Evaluation from corpus size perspective
\subsection{Discussion from Corpus Size Perspective}

% Warning!! Copy/Paste
In this section, we will consider the second part of our research question: How does the corpus size contribute to the embedding quality?

% Warning!! Copy/Paste
When we observe both accuracy and F1-Score measures from Fig.~\ref{fig:accuracy_plot} and \ref{fig:f1_plot}, it is clear the tendency for improvement while increasing corpus size. However, the metrics converge with the largest corpus sizes. There are two exceptions. The first one occurs with smaller values of corpus sizes for \textit{global} curve, as it decreases in F1-Score measures. The second corresponds to the last data point in \textit{air transport} curves. The former can happen we the classifier performs poorly for some classes while gets better in others. The latter may indicate that those curves could improve if we had more significant corpus sizes related to that context.

% Warning!! Copy/Paste
In general, we can note that the greater the corpus size in embeddings training, the better are the results. However, this impact decreases as the corpus size increases until a point where more words in the corpus have little impact on the results. %On the other hand, in the smaller corpus, the impact of increasing them is higher in relation to the greater corpus.
\section{Text Regression in Legal Judgments}

%----------------------------------------------------------
\section{Conclusions for the Chapter}



% ----------------------------------------------------------
% ----------------------------------------------------------
\chapter{Final Remarks} \label{cap:final_remarks}
% ----------------------------------------------------------

In this chapter, we conclude this work by recalling the research question, the objective, and the steps for their achievement. We also present the contributions from this research. Beyond contributions, it is highlighted the limitations faced during the executions of this research. Finally, there is the description of possibilities of future work to further explore and improve the experiments and results achieved as well as perspective of relevant research in the areas of study.

% ----------------------------------------------------------
\section{Conclusions}
% ----------------------------------------------------------


% Voltar para o objetivo
This work proposes the application of \gls{TM} and \gls{ML} techniques in legal judgments from the  \gls{JEC} at \gls{UFSC} to predict the possible results and the amount of compensation of immaterial damages. Thus, the research aims at contributing to the increase of agreements in conciliation hearings in the \gls{JEC} at \gls{UFSC}.

It is noteworthy that the conclusions presented in this work  are limited to judgments regarding failures in air transport services judged in the \gls{JEC} of \gls{UFSC}. Thus, the use of the \gls{ML} models trained in this work are limited to that court and that context. However, the word embeddings models are the exception, as they are trained on texts from several courts.

The research question from this work relates to whether it is possible to predict the result of a legal judgment based on its content and predict the amount of compensation for immaterial damage using \gls{ML} and \gls{TM} techniques. To answer the question, it is  divided into three parts, considering three distinct \gls{ML}  tasks: representation, classification, and regression.



Regarding the question on representation, the research evaluates the (in)existence of an impact of the specificity and the size of the corpora, used to train word embeddings,  in the performance of a text classification task. 
%We collect a significant amount of unlabeled text from higher courts in Brazil and texts related to miscellaneous contexts. We have thus three distinct contexts: air transport, general, and global, regarding legal judgments on air transport, general legal subjects, and miscellaneous contexts, respectively. Those corpora are divided into smaller subsets, which we use to train distinct GloVe embeddings representations. After applying them to the classification task, we observe that specificity and size matter until a certain point. 

We notice an improvement in having word embeddings trained with texts similar to those used in the classification task. There is also an improvement on performance when increasing the corpus size, however until a certain point.
We concluded that it is not required to have a corpus with billions of tokens for embeddings training to achieve good results in the classification of judgments from \gls{JEC}. A corpus with 100 million tokens related to air transport produced the best results in our experiments.

Concerning the question about classification, we evaluate and compare the performance of Classical \gls{ML} and \gls{DL} techniques in the classification of judgments from \gls{JEC} at \gls{UFSC}. 
%We divide the experiments into two: one using the complete judgments' text and another with the same judgments but removing the result part. 
%We apply several Classical \gls{ML} techniques with \gls{BOW} representation and three \gls{DL} techniques with several word embeddings trained with corpora related to air transport judgments. 
% As result, we observe in the experiments with full judgments' text that \gls{CNN} with Glove representation achieved the best performance in terms of accuracy. In the second part of the experiments we observed Classical \gls{ML} techniques performed better, that is Random Forest, Neural Networks, and Logistic Regression. 
We conclude that the application of the judgments prediction in the \gls{JEC} at \gls{UFSC} requires the use of the text from the facts and the applicable law, without results, i.e., without the final judgment. Only the facts and applicable law are available in the early stages of the legal case. Therefore, the use of Classical \gls{ML} techniques yield the best results. However, in the case of applying text classification for to organize the judgments according to their labels, using the complete judgments, \gls{CNN} with Glove embeddings would bring the best performance.

About the question on regression, we aim at evaluating whether the prediction of compensation for immaterial damage can be accurate and helpful in the legal environment using regression techniques. 
From the testing of several pipelines, we discovered that the adjustments \textit{N-grams Extraction} and \textit{Addiction of \gls{AELE}} have the biggest impact on prediction quality, while \textit{Feature selection}, \textit{Cross-validation} and \textit{Overfitting Avoidance} impact the execution time. 
%Another conclusion from the experiments relates to the fact the full pipeline does not bring the best prediction quality, i.e., there are other pipelines with better results. 
Finally, based on the evaluation from the legal expert, the best results for MAE are accurate in terms of compensation for immaterial damage and can be helpful in the conciliation hearings as they may encourage agreements between the parties.

In response to the research question, we conclude that it is possible to predict the judgment's outcome based on the text without the  result part, as the best classifier, Random Forest, achieved an accuracy of 82,2\%. 
As for the prediction of the amount of compensation for immaterial damage, it is possible to achieve accurate results  when using the set of the proposed adjustments and the regression techniques. The prediction quality achieved with them is acceptable, which facilitates the application in conciliation hearings.


We can also highlight some lessons learned during the execution of this research.
Most of them relate to the subjects of study, that is, Text Mining and Machine Learning, as the researcher and the legal expert had to learn the theory, the techniques, and their implementations from the beginning. Furthermore,  the researcher had the opportunity to adventure in a distinct area, the legal domain. This new knowledge will also be helpful in the daily life. In terms of acquired knowledge to the researchers, this work was very fruitful.  

% Importância do trabalho em grupo.
Another important lesson from this research is the team work. The experiments, papers, results and discussion presented here  are the product of a joint work between a master's student at PGEAS (the researcher), a doctoral student in law (the legal expert), and their advisors. Contributions also came from the \gls{EGOV} research group. The experience of working with researchers from distinct areas brought many opportunities of sharing methods, ideas, knowledge and others. 


\section{Contributions}
% ----------------------------------------------------------

During the research, the experiments conducted to answer the research questions resulted in publications in journals and conferences, as follows:


\begin{flushleft}

SABO, Isabela Cristina; DAL PONT, Thiago Raulino; ROVER, Aires José; HÜBNER, Jomi Fred. Classificação de sentenças de Juizado Especial Cível utilizando aprendizado de máquina. \textbf{Revista Democracia Digital e Governo Eletrônico}, v. 1, n. 18, p. 94–106, 2019.

\vspace{1em}

DAL PONT, Thiago Raulino; SABO, Isabela Cristina; HÜBNER, Jomi Fred;ROVER, Aires José. Impact of Text Specificity and Size on Word Embeddings Performance: An Empirical Evaluation in Brazilian Legal Domain. In: CERRI, Ricardo; PRATI, Ronaldo C (Eds.) \textbf{Intelligent Systems}. Cham: Springer International Publishing, 2020. p. 521–535. DOI: 10.1007/978-3-030-61377-8\_36.

\vspace{1em}

SABO, Isabela Cristina; DAL PONT, Thiago Raulino; WILTON, Pablo Ernesto Vigneaux; ROVER, Aires José; HÜBNER, Jomi Fred. Clustering of Brazilian legal judgments about failures in air transport service: an evaluation of different approaches. \textbf{Artificial Intelligence and Law}, Springer Netherlands, n. 0123456789, p. 1–37, Apr. 2021. ISSN 0924-8463. DOI:10.1007/s10506-021-09287-3.

\vspace{1em}

DAL PONT, Thiago Raulino; SABO, Isabela Cristina; HÜBNER, Jomi Fred; ROVER, Aires José. Regression applied to legal judgments to predict compensation for immaterial damage. \textbf{Natural Language Engineering}, 2021. \textbf{(Prelo)}
 
\end{flushleft}

The implemented code for each the publications
is freely available in repositories, as follows:

\begin{itemize}[noitemsep]
   
    \item Experiments regarding Text Classification with Classical \gls{ML} and \gls{DL} (\url{https://github.com/thiagordp/text_classification_in_legal_docs})
    \item Experiments regarding Word Embeddings in Portuguese (\url{https://github.com/thiagordp/embeddings_in_law_paper})
     \item Experiments regarding Clustering (\url{https://github.com/thiagordp/clustering_jec})
    \item Experiments regarding Text Regression  (\url{https://github.com/thiagordp/text_regression_in_law_judgments})

\end{itemize}



% Contribuições mais acadêmicas
Considering the published works and the experiments applied, the contributions to the state of the art in \gls{ML} applied to legal texts are as follows:

\begin{itemize}[noitemsep]
    \item Pre-trained word embeddings models for Brazilian legal texts, as there were no representations available until then.
    \item New application of regression in legal textual data.
    \item Impact of adjustments in the pipeline for regression in judgments from \gls{JEC} at \gls{UFSC}.
    \item Performance of Classical \gls{ML} and \gls{DL} techniques in the classification of legal judgments from \gls{JEC}.
\end{itemize}


% Contribuições mais práticas
In terms of practical contributions, the models and pipelines from this work may be adapted for the application in real legal conciliation hearings in the \gls{JEC} at \gls{UFSC}. A legal expert may present and explain the predictions to the parties. Thus, we expect to help them on reaching an agreement without the need of waiting for a judgment. In this way, the litigation in the \gls{JEC} would decrease, contributing to faster and more efficient access for citizens to justice.



\section{Limitations}

The first limitation concern the difficulties to gather the dataset for the experiments. All the legal judgments had to be collected manually by the legal expert into the \gls{JEC}. This is due to the fact that we wanted to avoid repeated judgments or judgments about a subject not fully related to failures in air transport service. We have the support of the current \gls{JEC}/\gls{UFSC} judge in this step. Although the Brazilian Judiciary indexes its processes according to the subject, there were changes in procedural electronic systems in this period, and the indexation may be incorrect due to human error. This is due to the fact that the lawsuit can be filed by different operators such as lawyers, consumers themselves, or by Judiciary employees. Efforts to unify data management in the Brazilian Judiciary are  recent, but there is still difficulties to access data for experiments. If the datasets were unified and available under proper request, the amount of work for this research would be considerably reduced.

Another limitation for the dataset relates to possibility of sharing the data in public datasets platforms. The current law does not allow the sharing of personal data contained in the legal judgments. However, it would be interesting to create mechanisms to allow datasets sharing, keeping due care with third-party data, and allowing the experiments to be reproducible for other researchers who read the published articles.  

In terms of predictions, although we achieved good results in classification and regression, the models do not provide the reasons that implied their predictions. When applying the models in a real conciliation hearing, the parties would ask how did the system come to this decision?  Thus, there is the necessity of providing explanation to predictions from ML models.



\section{Future Work}

% Bigger dataset for DL.

Considering the application of ML in the legal judgments from \gls{JEC}, a possible improvement relates to the dataset size increasing. A larger dataset would allow future experiments regarding \gls{DL} tasks that were not explored in this research, such as text summarizing, generation and others.  Two possible paths to do that are the uses of Variational Auto-Encoders  and Generative Adversarial Networks  \cite{Kingma2019, Iqbal2020}.

In terms of text representation, new neural based representation techniques are often proposed in the literature, each of which having more and more trainable parameters. Some examples of representation techniques include: \gls{GPT-3} \cite{Brown2020}, \gls{ELMo} \cite{Peters2018}, \gls{BERT} \cite{Devlin2018} and others.
Thus, a first improvement in the text representation proposed in this research is the training of models for these newer techniques using the collected corpora. However, training these representations may take longer.
Furthermore, considering the complexity of the written language used in the legal domain, it may yield relevant contributions the use of sophisticated  representation techniques in applications like automatic judgments' generation or the \textit{translation} of the legal text to a simpler writing, understandable to the common citizen.

Regarding the prediction of legal judgments, both classification and regression, an important  improvement is the use of legal documents from the early stages of a lawsuit, that is, the  complaint from the client, the arguments from the air company, and others. Considering the described limitations of acquiring this type of data, the presented experiments involved only judgments with the summary of the arguments from the parties, and the applicable law. 


Despite time constraints to finish this work, in the future, we would like to adapt the pipelines used in the regression experiments to the classification task. The pipelines for the classification task using Classical ML techniques  were similar to the \textit{baseline} for regression. Thus, there is a possibility of improvements by applying the adjustments in the pipelines for classification.

% Entender porque SVM não deu certo de jeito nenhum
% A ideia de um sistema composto de duas partes
% Explainability

Another improvement relates to how the users see the predictions from the ML models, that is, the interpretation of the decisions. The current implementation of the pipelines and techniques does not allow the user to understand the steps taken to produce the final prediction. A possible solution to overcome such limitation is the concept of \gls{XAI} \cite{Tjoa2019, Bibal2020}. For example, considering the practical use case in a conciliation hearing. The legal expert would introduce the system and communicate its predictions on the case at hand. However, the parts could ask how the systems got such prediction. Using \gls{XAI} enhancements, the system would highlight the relevant facts, the applicable law, the previous decisions in similar cases, and  finally, its predictions. 



% ----------------------------------------------------------
% ELEMENTOS PÓS-TEXTUAIS
% ----------------------------------------------------------
\postextual
% ----------------------------------------------------------

% ----------------------------------------------------------
% Referências bibliográficas
% ----------------------------------------------------------
\begingroup
    \printbibliography[title=REFERENCES]
\endgroup

% ----------------------------------------------------------
% Glossário
% ----------------------------------------------------------
%
% Consulte o manual da classe abntex2 para orientações sobre o glossário.
%
%\glossary

% ----------------------------------------------------------
% Apêndices
% ----------------------------------------------------------

% ---
% Inicia os apêndices
% ---
\begin{apendicesenv}
%	\partapendices* 
	% ----------------------------------------------------------
\chapter{Systematic Review of the Literature: AI, ML and Law} \label{ap:rsl_ml_law}
% ----------------------------------------------------------

The Systematic Review of the Literature, detailed in this section, focused on finding works related to the applications of TM and ML in the legal domain.


\section{Definition of Search Questions}

The main question in this SRL is: ``Which are the Text Mining Techniques applied in the legal domain?''

As secondary questions, there is:

\begin{itemize}[noitemsep]
    \item Which is the legal application?
    \item What are the pre-processing techniques used?
    \item What are the representation techniques used?
    \item What are the feature extraction techniques used?
    \item What are the classification techniques used?
    \item What are the clustering techniques used?
    \item What are the regression techniques used?
    \item What are the evaluation techniques used?
\end{itemize}

\section{Search Strategies}

In order to have an overview of the publications regarding TM and legal domain, we searched in February 29, 2020, for works published on conferences or journals from 2010 to 2020.  The search used the following search string:

\begin{verbatim}
    ("text mining" OR "natural language processing" OR "nlp" OR
    "language processing") AND ("deep learning" OR "machine learning")
    AND ("classification" OR "cluster*" OR "regression" OR 
    "categorization" OR "embedding*" OR "representation" OR "predict*")
    AND ("text" OR "document") AND ("law" OR "legal" OR "judicial" OR
    "justice" OR "court" OR "legislation" OR "juridical" OR "lawful")
\end{verbatim}

In this search, we also added synonymous for Machine Learning, Text Mining, and ML Tasks, and the legal domain.

The search embraced the papers' titles, abstracts and keywords.

\section{Knowledge Bases}

In this SRL, we included bases predominantly related to computing as well as interdisciplinary basis.  The list of knowledge bases follows:

\begin{itemize}[noitemsep]
    \item Scopus
    \item IEEE Xplore
    \item Web of Science
    \item ACM Digital Library
\end{itemize}

\section{Inclusion and Exclusion Criteria}

Following the questions of this research, we defined a set of inclusion and exclusion criteria. The process of selection embraced the reading of title, abstract and keywords and the accordance with the criteria:

The following is the list of inclusion criteria:

\begin{itemize}[noitemsep]
    \item Published in journal or conference
    \item Involves legal processes, or texts with juridical language;
    \item Involves Machine Learning or Text Mining;
    \item The techniques used are named;
    \item Empirical Works.
    \item Published from 2010 and February 2020.
\end{itemize}

And the following is the list of exclusion criteria:

\begin{itemize}[noitemsep]
    \item Not written in Portuguese or English;
    \item Published over 10 years ago;
    \item Does not involve Machine Learning or Text Mining;
    \item Does not involve the legal domain;
    \item Theoretical works.
\end{itemize}


\section{Data Extraction Plan}

Information to extract from the papers:

\begin{itemize}[noitemsep]
    \item Title
    \item Keywords
    \item Abstract
    \item Authors
    \item Year of publication
    \item Journal or Conference
    \item Authors affiliation
    \item Pre-processing techniques
    \item Representation / Feature Extraction Techniques
    \item Classification  / Clustering / Regression Techniques
    \item Other techniques
    \item Model Evaluation Techniques
\end{itemize}


\section{Search Execution and Preliminary Analysis}

After applying the search strings to the knowledge bases on February 29, 2020 without filters on year of publication, we obtained the number of publications shown in Figure \ref{fig:ap_rsl_tm_law}.

\begin{figure}[H]
    \centering
    \caption{Publications on ML and TM applied to the legal domain without year filtering}
    \label{fig:ap_rsl_tm_law}
    \includegraphics[width=\textwidth]{images/appendix/artigos_tm_law.png}
\end{figure}

After applying the search strings to the knowledge bases with time filtering on February 29, 2020, the search returned 195. After duplicates removal, the number of works reduced to 147. Finally, from the reading of title, abstract and keywords and the application of the selection criteria, the number of works selected for full reading reduced to 46.

The selected works were used as references in the Introduction chapter, in the Related Works and in the Background.

\section{Results and Analysis}
In this section we show a sequence of quantitative data in terms of the tens most frequent authors, journals, text mining techniques and others.

In the following tables, there is ten most frequent authors and conferences found in the SRL and the most frequent techniques for pre-processing, representation, classification, clustering and evaluation.

% Legal Applications?

% Authors

\begin{table}[H]
    \centering
    \caption{Ten most frequent authors}
    \label{tab:rsl_freq_authors}
    \begin{tabular}{@{}cc@{}}
    
    \toprule
    \textbf{Authors}     & \textbf{Papers} \\ \midrule
    Matthes, Florian     & 5               \\
    Glaser, Ingo         & 4               \\
    Chalkidis, Ilias     & 3               \\
    Scepankova, Elena    & 3               \\
    Quaresma, Paulo      & 2               \\
    Gonçalves, Teresa    & 2               \\
    Galgani, Filippo     & 2               \\
    Compton, Paul        & 2               \\
    Hoffmann, Achim      & 2               \\
    Androutsopoulos, Ion & 2               \\ \bottomrule
    
    \end{tabular}
\end{table}

% Journals / Conferences

\begin{table}[H]
    \centering
    \caption{Ten most frequent journals and conferences}
    \label{tab:rsl_freq_conferences}
\begin{tabular}{@{}p{14cm}c@{}}
\toprule
\multicolumn{1}{c}{\textbf{Journal / Conference}}                                                         & \textbf{Papers} \\ \midrule
Lecture Notes in Computer Science                                                                         & 8               \\
CEUR Workshop Proceedings                                                                                 & 4               \\
Frontiers in Artificial Intelligence and Applications                                                     & 3               \\
Artificial Intelligence and Law                                                                           & 2               \\
2010 6th International Conference on Wireless Communications, Networking and Mobile Computing, WiCOM 2010 & 1               \\
Proceedings of the ACM Conference on Computer and Communications Security                                 & 1               \\
Foundations and Trends in Information Retrieval                                                           & 1               \\
Conference on Legal Knowledge and Information Systems                                                     & 1               \\
Expert Systems with Applications                                                                          & 1               \\
International Conference on Cloud Computing and Services Science                                          & 1               \\ \bottomrule
\end{tabular}
\end{table}

% Pré-processing Techniques


\begin{table}[H]
\centering
\caption{Ten most frequent pre-processing techniques}
\label{tab:rsl_freq_pre_processing}

    \begin{tabular}{@{}ll@{}}
    
    \toprule
    \multicolumn{1}{c}{\textbf{Preprocessing}} & \multicolumn{1}{c}{\textbf{Papers}} \\ \midrule
    Stop words removal                         & 15                                  \\
    Lemmatization                              & 8                                   \\
    Stemming                                   & 7                                   \\
    Tokenization                               & 4                                   \\
    Lowercase                                  & 4                                   \\
    POS Tagging                                & 3                                   \\
    Normalization                              & 3                                   \\
    Remove punctuation                         & 3                                   \\
    Remove noise                               & 1                                   \\
    Regularization                             & 1                                   \\ \bottomrule
    
    \end{tabular}
\end{table}


% Representation Techniques
\begin{table}[H]
\centering
\caption{Ten most frequent representation techniques}
\label{tab:rsl_freq_representation}
\begin{tabular}{@{}lc@{}}
\toprule
\textbf{Representation}  & \textbf{Papers} \\ \midrule
TF-IDF                   & 19              \\
Word2Vec                 & 14              \\
Bag of Words             & 10              \\
N-Gram                   & 7               \\
Part-of-Speech Tag       & 6               \\
Named Entity Recognition & 5               \\
Doc2Vec                  & 4               \\
Word Embeddings          & 3               \\
FastText                 & 3               \\ 
BERT                     & 2               \\ \bottomrule
\end{tabular}
\end{table}

% Classification techniques
\begin{table}[H]
\centering
\caption{Ten most frequent classification techniques}
\label{tab:rsl_freq_classification}
\begin{tabular}{@{}cc@{}}
\toprule
\textbf{Classification Tech} & \textbf{Papers} \\ \midrule
Support Vector Machine       & 24              \\
Convolution Neural Network   & 19              \\
Naïve Bayes                  & 18              \\
Decision Tree                & 17              \\
k Nearest Neighbors          & 14              \\
Recurrent neural network     & 14              \\
Random Forest                & 13              \\
Long Short Term Memor y      & 13              \\
Logistic Regression          & 11              \\
Conditional Random Field     & 7               \\ \bottomrule
\end{tabular}
\end{table}


% Clustering Techniques
\begin{table}[H]
\centering
\caption{Most frequent clustering techniques}
\label{tab:rsl_freq_clustering}
\begin{tabular}{cc}
\hline
\textbf{Clustering Tech} & \textbf{Papers} \\ \hline
Hierarchical Clustering  & 2               \\
Fuzzy C-Means            & 1               \\
Hierarchical LDA         & 1               \\
k-Means                  & 1              
\end{tabular}
\end{table}

\begin{table}[H]
\centering
\caption{Ten most frequent Evaluation Metrics}
\label{tab:rsl_freq_evaluation}
\begin{tabular}{cc}
\hline
\textbf{Evaluation Metric} & \textbf{Papers} \\ \hline
F1-score                   & 23              \\
Accuracy                   & 21              \\
Precision                  & 20              \\
Recall                     & 19              \\
Cross-validation           & 6               \\
Rouge                      & 2               \\
Area under curve ROC       & 2               \\
ROC                        & 1               \\
BLEU                       & 1               \\ \hline
\end{tabular}
\end{table}



% Regression Techniques
This SRL did not find works that applied regression techniques in the legal domain.





	% ----------------------------------------------------------
\chapter{Systematic Review of the Literature: Text Representation}\label{ap:rsl_representation_law}
% ----------------------------------------------------------


% - Focus on word embeddings and language models
% - 

In this SRL, we tried to find work related to the application of text representation techniques on legal texts written in the Portuguese language. However, such task did not succeed due to the absence of work in that sense. Thus, two smaller SRLs were conducted to find papers with broader searches. The first focused on the representation of legal texts in any language and the second in the representation of general texts in Portuguese. 

\section{Definition of Search Questions}

The question of the first search: ``What are the text representation techniques applied to texts from the legal domain?''

The question of the second search: ``What are the text representation techniques applied to texts written in Portuguese?''

\section{Search Strategies}

Search for representation of legal texts in any languages:

\begin{verbatim}
("legal" OR "law" OR "court" OR "justice") AND ("embedding*" OR 
"language model*" OR "machine learning" OR "deep learning" OR 
"natural language processing" OR "text mining") AND ("doc2vec" 
OR "paragraph2vec" OR "word2vec" OR "glove" OR "wang2vec" OR 
"fasttext" OR "bert" OR "elmo" OR "law2vec")
\end{verbatim}


Search for representation of general texts in Portuguese:

\begin{verbatim}
("portuguese" OR "brazil*") AND ("embedding*" OR "deep learning" OR 
"machine learning" OR "natural language processing" OR "text mining") 
AND ("doc2vec" OR "paragraph2vec" OR "word2vec" OR "glove" OR 
"wang2vec" OR "fasttext" OR "bert" OR "elmo" OR "law2vec" ) 
\end{verbatim}



\section{Knowledge Bases}

In this SRL, we searched on the following bases:

\begin{itemize}[noitemsep]
    \item Scopus
    \item ACM Digital Library
    \item IEEE Xplore
    \item Web of Science
\end{itemize}


\section{Inclusion and Exclusion Criteria}


Following the questions of this research, we defined a set of inclusion and exclusion criteria. The process of selection embraced the reading of title, abstract and keywords and the accordance with the criteria:

The following is the list of inclusion criteria:

\begin{itemize}[noitemsep]
    \item Published in journal or conference
    \item Involves legal texts in any languages or general texts in Portuguese;
    \item Involves Machine Learning or Text Mining;
    \item The techniques used are named;
    \item The work evaluate or train representations
    \item Empirical Work;
    \item Published from 2010 and May 2020.
\end{itemize}

And the following is the list of exclusion criteria:

\begin{itemize}[noitemsep]
    \item Work not written in Portuguese or English;
    \item Published over 10 years ago;
    \item Does not involve legal texts in any language neither general texts in Portuguese;
    \item Does not involve Machine Learning or Text Mining;
    \item Theoretic work
\end{itemize}

\section{Data Extraction Plan}

In the SRL from this section, we focused on just retrieving the representation techniques used and the application.

\section{Search Execution and Preliminary Analysis}

After applying the first search string to the knowledge base on May 7, 2020 for researches published from 2010 to May 2020, the search returned 52 documents. After reading title, abstract and keywords and applying the selection criteria, the number of papers reduced to 12.

In terms of the second search string, after applying to the knowledge base on May 7, 2020 with the same time filtering, the search returned 136 documents. After reading title, abstract and keywords and applying the selection criteria, the number of papers reduced to 20.


\section{Results and Analysis}

In this section, we show the results and analysis in terms of representation techniques for the first part of this SRL related to legal texts from many languages and general texts in Portuguese.

In the first part of this research, we focused on the representation techniques applied in the legal domain. In Figure \ref{fig:rsl_legal_representation}, one can see the distribution by year of the research interest, on the area of representation of legal documents for ML tasks, considering our selection criteria. Although we set the interval to ten years, the selected work only embraced three distinct years.


\begin{figure}[htb]
    \centering
    \caption{Researches by year for text representation in legal documents}
    \label{fig:rsl_legal_representation}
    \includegraphics[width=\textwidth]{images/appendix/rsl_legal_representation.png}
\end{figure}

In Table \ref{tab:rsl_representation_legal}, there is the list of representation techniques used in the selected work. Note that, many papers reported using more than one representation techniques in their experiments.

% Most frequent Representations
\begin{table}[htb]
\centering
\caption{Representations applied to legal texts}
\label{tab:rsl_representation_legal}
\footnotesize
\begin{tabular}{cc}
\hline
\textbf{Representation} & \textbf{Papers} \\ \hline
Word2Vec                & 8               \\
Doc2Vec                 & 2               \\
WordVec CBOW           & 2               \\
Glove                   & 2               \\
FastText                & 2               \\
ELMo                    & 1               \\
Bag of Words            & 1               \\
Law2Vec                 & 1              \\\bottomrule
\end{tabular}
\end{table}

In the second part of this research, we focused on the representation techniques applied in general texts written in the Portuguese language. In Figure \ref{fig:rsl_representation_portuguese_year}, one can see the distribution by year of the research interest on the area, considering our selection criteria. Although, the SRL embraced the last ten years the selected work embraced the last five.


\begin{figure}[htb]
    \centering
    \caption{Researches by year for text representation in Portuguese documents}
    \label{fig:rsl_representation_portuguese_year}
    \includegraphics[width=\textwidth]{images/appendix/rsl_portuguese_representation.png}
\end{figure}

In Table \ref{tab:rsl_representation_portuguese}, there is the list of representation techniques used in the selected work. Note that, many papers reported using more than one representation techniques in their experiments.

\begin{table}[htb]
\centering
\caption{Representation used in Portuguese texts}
\label{tab:rsl_representation_portuguese}
\footnotesize
\begin{tabular}{cc}
\hline
\textbf{Representation} & \textbf{Papers} \\ \hline
Word2Vec                & 7               \\
Word2Vec Skipgram       & 7               \\
Glove                   & 6               \\
TF-IDF                  & 3               \\
Wang2Vec Skipgram       & 3               \\
Bag of Words            & 2               \\
ELMo                    & 2               \\
FastText                & 2               \\
FastText Skipgram       & 2               \\
LDA                     & 2              \\ \bottomrule
\end{tabular}
\end{table}

% Tasks

As mentioned, we do not find, until the date of the SRLs, any work related to evaluation or training of representations of  legal texts in the Portuguese language.


	% ----------------------------------------------------------
\chapter{Systematic Review of the Literature: Text Regression}\label{ap:rsl_regression_law}
% ----------------------------------------------------------

In this SRL, we focused on finding papers related to the application of regression techniques on legal texts written in the Portuguese language. However, such a task did not succeed due to the lack of work in that matter. Thus, we applied a broader search regarding the application of regression in texts without context or language limitations.

\section{Definition of Search Questions}

The question of this search is: ``What are the regression techniques applied to texts considering any applications and languages?''

\section{Search Strategies}

The search used the following search string:

% TODO: Continue from here.
\begin{verbatim}
  ( "regression on text*"  OR  "text* regression"  OR  "regression text*"  
  OR  "regression from text*"  OR  "regression for text*" )  
  AND NOT  "logistic regression"  
\end{verbatim}



\section{Search Resources}
In this SRL, we searched on the following bases:

\begin{itemize}[noitemsep]
    \item Scopus
    \item ACM Digital Library
    \item IEEE Xplore
    \item Web of Science
\end{itemize}

\section{Selection Criteria}

Following the questions of this research, we defined a set of inclusion and exclusion criteria. The process of selection embraced the reading of title, abstract and keywords, and the accordance with the criteria.

The following is the list of inclusion criteria:

\begin{itemize}[noitemsep]
    \item Published in journal or conference
    \item Involves regression applied to textual data;
    \item Involves Machine Learning or Text Mining;
    \item The techniques used are named;
    \item Empirical Work;
    \item Published from 2010 and December 2020.
\end{itemize}

And the following is the list of exclusion criteria:

\begin{itemize}[noitemsep]
    \item Work not written in Portuguese or English;
    \item Published over 10 years ago;
    \item Does not involve regression applied to textual data;
    \item Does not involve Machine Learning or Text Mining;
    \item Theoretic work.
\end{itemize}

\section{Data Extraction}
In the SRL from this section, we focused on retrieving the text representation, the regression techniques used and the evaluation metrics.

\section{Search Execution and Preliminary Analysis}

After applying the first search string to the knowledge base on December 1, 2020 for researches published from 2010 to December 2020, the search returned 124 documents. After reading title, abstract and keywords and applying the selection criteria, the number of papers reduced to 18.


\section{Results and Discussion}

% Técnicas mais comuns

In this section, we show the results and analysis in terms of regression techniques applied to text. In this part of this research, we focused on the regression techniques applied in any domains. In Figure \ref{fig:rsl_regression_year_publishing}, one can notice the distribution by year of the research interest on the area of the regression applied to texts.

% Publicações por ano

\begin{figure}[htb]
    \centering
    \caption{Papers published by year}
    \label{fig:rsl_regression_year_publishing}
    \includegraphics[width=\textwidth]{images/appendix/rsl_regression.png}
\end{figure}


In Table \ref{tab:rsl_regression_representation}, there are the representation techniques used in the selected work from the SRL. The most common technique is the Bag of Words followed by the N-Grams and TF-IDF, that is Vector Space Models representations. There is also neural based techniques such as word embeddings.


\begin{table}[htb]
\centering
\caption{Representation techniques in papers from regression SRL}
\label{tab:rsl_regression_representation}
\footnotesize
\begin{tabular}{@{}cc@{}}
\toprule
\textbf{Representation} & \textbf{Papers} \\ \midrule
Bag of Words & 8 \\
N-gram & 4 \\
TF-IDF & 4 \\
Metadata & 2 \\
Part-of-Speech Tag & 2 \\
Word Embeddings & 2 \\
Context ependence & 1 \\
Dependency relations & 1 \\
LDA & 1 \\
LSA & 1 \\ \bottomrule
\end{tabular}
\end{table}


In Table \ref{tab:rsl_regr_tech}, there are the most frequent regression techniques applied in the selected work. The most common is the Support Vector Machine, followed by Linear Regression, Convolution Neural Network, Elastic Net, Gaussian Copula, and Gradient Boosting.

\begin{table}[htb]
\centering
\caption{Regression techniques applied in the  papers from SRL}
\label{tab:rsl_regr_tech}
\footnotesize
\begin{tabular}{@{}lc@{}}
\toprule
\multicolumn{1}{c}{\textbf{Regression Tech}} & \textbf{Papers} \\ \midrule
Support Vector Machine & 7 \\
Linear Regression & 5 \\
Convolutional Neural Network & 2 \\
Elastic Net & 2 \\
Gaussian Copula & 2 \\
Gradient Boosting & 2 \\
Conditional Generative Adversarial Network & 1 \\
Gaussian Process & 1 \\
kNN & 1 \\
Lasso & 1 \\
Multinomial logistic text regression & 1 \\
Random Forest & 1 \\
Ridge & 1 \\
XGBoosting & 1 \\ \bottomrule
\end{tabular}
\end{table}

In Table \ref{tab:rsl_regr_evaluation}, there are the most common evaluation metrics for regression applied in the select work. One can note the Mean Absolute Error (MAE) is the most common, followed by Mean Square Error (MSE) and Root Mean Square Error (RMSE).

\begin{table}[htb]
\centering
\caption{Evaluation metrics for regression}
\label{tab:rsl_regr_evaluation}
\footnotesize
\begin{tabular}{@{}lc@{}}

\toprule
\multicolumn{1}{c}{\textbf{Evaluation Metric}} & \textbf{Papers} \\ \midrule
Mean Absolute Error & 9 \\
Mean Square Error & 3 \\
Root Mean Square Error & 3 \\
Pearson's correlation & 2 \\
Relative Absolute Error & 2 \\
Root Relative Squared Error & 2 \\
Adjusted R2 & 1 \\
F-variation & 1 \\
Kendall's Tau & 1 \\
R$^2$ & 1 \\
RAE & 1 \\
Spearman's Correlation & 1 \\
Standardization on Beta & 1 \\
Symmetric Mean Absolute Percentage Error & 1 \\
Value of $t$ & 1 \\ \bottomrule
\end{tabular}
\end{table}


\end{apendicesenv}
% ---


% ----------------------------------------------------------
% Anexos
% ----------------------------------------------------------

% ---
% Inicia os anexos
% ---
\begin{anexosenv}
%	\partanexos*
	\input{aftertext/anexo_a}
\end{anexosenv}

%---------------------------------------------------------------------
% INDICE REMISSIVO
%---------------------------------------------------------------------
%\phantompart
%\printindex
%---------------------------------------------------------------------

\end{document}
